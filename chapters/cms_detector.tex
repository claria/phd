% vim: tw=80

\chapter{Experimental Setup}
\label{sec:experimental_setup}

The European Organization for Nuclear Research (\CERN) was founded in 1954.
Originally dedicated to the study of atomic nuclei, it is now devoted to the
research of sub-atomic particles and their interactions. To accomplish that
task, \CERN built several particle accelerators reaching record-breaking
energies and explored energy ranges not accessible before.

Ground-breaking achievements like the discovery of neutral currents in the
Gargamelle bubble chamber~\cite{Hasert:1973ff}, the discovery of the W and Z
boson by the UA1 and UA2 experiments~\cite{Arnison:1983rp} at the SPS
accelerator or the creation of antihydrogen atoms were made by physicists at
\CERN.

Remarkable insights into the Standard Model were gained by measurements at the
LEP accelerator at which the mass of the of the Z~boson and W~boson were
precisely measured. The proton-antiproton collider Tevatron at FNAL discovered
and measured the mass of the top quark accurately. Since the search for the long
anticipated higgs boson was unsuccesful due to their limited energy reach, an even
larger and more powerful accelerator was planned and built, the Large Hadron
Collider (LHC). The search for the higgs boson finally succeded in
2012~\cite{Chatrchyan:2012xdj,Aad:2012tfa}, but many further questions like the search for
supersymmetry or dark matter still need to be resolved in the future.

\section{The Large Hadron Collider}

The LHC is the world's most powerful particle accelerator and collider. The LHC
is contained in the circular tunnel of the preceding LEP collider which has a
circumference of \SI{27}{\kilo \meter} at a depth between \SI{50}{\meter} and
\SI{175}{\meter}.  The tunnel crosses the border between Switzerland and France
at four points and two of the four main experiments are located in France. 

Two adjacent beamlines that intersect at four interactions points contain the
particle beams travelling in opposite directions. More than 1000 dipole magnets,
generating a magnetic field of up to \SI{8.3}{\tesla} bend the beams on
a circular track while almost 400 quadrupole magnets keep the beams focused. 

The beams are brought to collision at four interaction points which house the
LHC experiments ALICE~\cite{ALICE}, ATLAS~\cite{ATLASa},
\CMS~\cite{Bayatian:922757,Ball:2007zza,Chatrchyan:2008aa} and \textsc{LHCb}~\cite{LHCb}. ALICE is
designed to study the quark-gluon plasma produced by colliding heavy ions, which
resembles the initial state of the universe. LHCB is precisely measuring the CP
violation and the decay of B mesons. ATLAS and CMS, general purpose detectors
which allow a broad field of physics studies, were built to search and study the
Higgs boson and physics models beyond the Standard Model. Furthermore, precision
measurements of Standard Model predictions and its parameters improve the
current knowledge and the confidence of its predictions.

Prior to injection and acceleration of protons in the LHC, the particles pass a
series of consecutive acceleration steps succesively increasing their energy.
The linear particle accelerator (LINAC2) generates \SI{50}{\mega \electronvolt}
protons, that are further accelerated in the Proton-Synchrotron (PS) to
\SI{26}{\giga \electronvolt} and the Super-Proton-Synchrotron (SPS) up to an
energy of \SI{450}{\giga \electronvolt}. At last the proton beams are injected
into the LHC ring and are further accelerated up to peak design energy of
\SI{13}{\tera\electronvolt}. All these pre-accelerators are not only used to
feed the LHC, but also serve other physics experiments li,e the radioactive ion
beam facility ISOLDE, see Fig.~\ref{fig:lhc_complex}.

\begin{figure}[htp]
    \centering
    \includegraphics[width=0.8\textwidth]{figures/cms_detector/lhc_accelerator_chain.pdf}
    \caption[\CERN accelerator complex]{Several particle accelerators are
        chained together to feed proton beams into the LHC. Further experiments are
        located along the accelerator complex serving a broad program of physics
        studies~\cite{LHC:COMPLEX}.}
    \label{fig:lhc_complex}
\end{figure}

\section{Luminosity measurement}
\label{sec:lumi_measurement}

The cross section $\sigma$ of a physical process is related to the
event rate $\dot{N}$ by the luminosity~$\mathcal{L}$

\begin{equation*}
    \dot{N} = \mathcal{L} \sigma.
\end{equation*}

The luminosity is dependent on the particle beam parameters and can be expressed
by

\begin{equation*}
    \mathcal{L} = = \frac{n_p^2
        n_b f_\mathrm{rev} \gamma F}{4 \pi \epsilon_n \beta^*},
\end{equation*}

where $n_p$ is the number of particles per bunch, $n_b$ is the number of bunches
per beam, $f_\mathrm{rev}$ the revolution frequency, $\gamma$ the relativistic
gamma factor and $F$ the geometric luminosity reduction factor. The effective
collision area of the two beams is related to the normalized transverse beam
emittance $\epsilon_n$ and the value of the betatron function $\beta^*$ at the
interaction point.

The instantaneous luminosity is measured constantly by the experiments. CMS
employs two methods to measure the relative instantaneous
luminosity~\cite{CMS-PAS-LUM-13-001}. The first method measures the particle
flux in the hadron forward calorimeter which is related to the instantaneous
luminosity. The second counts the number of clusters in the pixel tracking
detector measured in zero-bias events. The absolute luminosity measurement is
relying on van-der-Meer scans which are carried out in special runs of the
LHC~\cite{vanderMeer:296752}.

Fig.~\ref{fig:cms:lumi_integrated} shows the integrated luminosity delived by
LHC to the CMS experiment in the run periods from 2010 to 2012.

\begin{figure}[htp]
    \centering
    \includegraphics[width=0.9\textwidth]{figures/cms_detector/lumi_integrated.pdf}
    \caption[Integrated luminosity delivered to CMS]{Integrated luminosity
        delivered by the LHC to CMS in the 2010, 2011 and 2012 LHC run periods.
        Taken from~\cite{Berger:2014aca}.}
    \label{fig:cms:lumi_integrated}
\end{figure}


\section{The Compact Muon Solenoid Detector}

The Compact Muon Solenoid (CMS) detector is a general purpose detector at the
LHC, located at Point 5 of the LHC ring. To serve a wide range of physics
studies, the detector design is driven by a cylyndrical structure containing
layers of different subdetectors each built to measure a specific type of
particles with best precision, see Fig.~\ref{fig:cms:transverse_slice}. A
high-precision inner tracking system is surrounded by an electromagnetic and a
hadronic calorimeter which again are enclosed by a superconducting solenoid
magnet. The whole inner part of the detector is surrounded by a sophisticated
muon detection system embedded in a iron yoke, see
Fig.~\ref{fig:cms:longitudinal_section}. The detector is \SI{21.6}{\meter}
long and \SI{14.6}{\meter} in diameter, but weighing more than
\SI{12000}{\tonne} due to its compact design.  It was built as cylyndrical
slices constructed at ground level and lowered into the cavern. In case of
upgrades or repairs, the slices can be pulled apart and the inner components can
be easily accessed.

\begin{figure}[htp]
    \centering
    \includegraphics[width=0.9\textwidth]{figures/cms_detector/cms_slice.png}
    \caption[Transverse slice of the CMS detector]{The transverse slice through the
        CMS detector shows the various subdetectors and the signatures by which
        they are detected. Taken from~\cite{Barney:2120661}.}
    \label{fig:cms:transverse_slice}
\end{figure}


Fig.~\ref{fig:cms:longitudinal_section} shows a longitudinal section of one
quadrant of the CMS detector and the location of all subsystems.

The detector and physics performance of the CMS detector are discussed in great
detail in~\cite{Bayatian:922757,Ball:2007zza,Chatrchyan:2008aa}. This section
intends to only present a short overview of the design and functional principles
of the detector.

\begin{figure}[htp]
    \centering
    \includegraphics[width=1.0\textwidth]{figures/cms_detector/cms_longitudinal_section.pdf}
    \caption[Longitudinal section of the CMS
    detector]{A longitudinal section of one quadrant of the CMS
        detector in the $y$-$z$ plane. The sketch shows
    the multi-layer design of the CMS detector starting with the silicon pixel
and silicon strip detectors close to the interaction point. They are surrounded
by the electromagnetic (green) and hadronic (yellow) calorimeters. The barrel is
encompassed by the superconducting magnet (blue). The muon detection system
(red) is embedded in the iron return yoke~\cite{Berger:2014aca}.}
    \label{fig:cms:longitudinal_section}
\end{figure}

\subsection{Definition of the Coordinate System}

CMS uses a right-handed coordinate system centered at the nominal interaction point inside the
detector. The $x$-axis points radially inward towards the LHC ring centre, the
$y$-axis vertically upwards and the $z$-axis along the beam direction towards
the Jura mountains. Important quantities are the azimuthal angle $\phi$,
measured from the $x$-axis in the $x$-$y$ plane, and the polar angle $\theta$,
measured from the $z$-axis in the $z$-$y$ plane. Instead of the polar angle
$\theta$, the pseudo-rapidity $\eta$ and the rapidity $y$ are commonly used to
divide the phasespace. The pseudo-rapidity is defined as

\begin{equation*}
    \eta = - \ln \left( \tan \left( \frac{\theta}{2} \right) \right).
\end{equation*}

Throughout this thesis the rapidity is favoured compared to the pseudo-rapidity
due to its invariance under longitudinal boosts in the $z$-direction. Rapidity
and pseudo-rapidity are equivalent in the case of massless particles. The
rapidity is defined as

\begin{equation*}
    y = \frac{1}{2} \ln \left( \frac{E + p_z}{E - p_z} \right).
\end{equation*}

The momentum along the beamline is not well defined due to the momentum
distribution inside the proton. A direct connection to the hard process is given
by the transverse momentum \pt related to cartesian coordinates as

\begin{equation*}
    \pt = \sqrt{p_x^2 + p_y^2}.
\end{equation*}

\subsection{Inner Tracking System}

To yield a best possible spatial resolution, the particle tracks need to be
measured as close to the beam line as possible. The inner tracking system of CMS
consists of silicon detectors which measure the hits of charged
particles emerging from the collision. 

The silicon detectors are depledted from free charged by applying a voltage to
them. As charged particles pass through the detector material, they leave a
small ionization current which can be detected and measured as a hit in the
detector. By combining multiple hits, the track of a charged particle can be
reconstructed and the momentum and charge of the particle determined based on a
mass hypothesis. Due to the strong magnetic field of the CMS detector, even
tracks of particles with high transverse momenta have a measurable curvature.
 
The inner tracking detector encloses the interaction point with a diameter of 2.6 m and
extends up to 2.8 m in each direction along the beampipe, the tracking system covers a
pseudo-rapidity range up to $|\eta| < 2.4$. The inner tracking system comprises
two subsystems, the silicon pixel detector consisting of three layers and
installed very close to the beam pipe and the silicon strip detector with ten
strip layers in the barrel region, see Fig.~\ref{fig:cms:inner_tracking}. 

\paragraph{Silicon Pixel Detector} Containing over \SI{65} milion pixels
arranged in three cylindrical layers at \SI{4}{\centi\meter},
\SI{7}{\centi\meter} and \SI{11}{\centi\meter} distance to the beam pipe, the
pixel detector is able to resolve the tracks of the huge number particles. At
LHC design luminosity, about 1000 particles pass the tracking detector on
average. The size of each pixel is \SI{100}{\micro \meter} x \SI{150}{\micro
\meter} giving a average occupancy of $10^{-4}$ per bunch crossing.  The high
spatial resolution achieved by the pixel detector furthermore allows the
identification and measurement of secondary vertices used to identify long-lived
particles.

\paragraph{Silicon Strip Detector} The pixel detector is complemented by a silicon
strip detector. Reduced particle flux further away from the beam pipe eases the identification
of tracks. Cost-efficient silicon strips are employed reaching out to
a radius of \SI{1.3}{\meter}. The strip detector consists of a total of 10 million
detecting strips read out by \SI{80000} chips and are arranged overlapping to
avoid any blind detector area.

\begin{figure}[htp]
    \centering
    \includegraphics[width=0.65\textwidth]{figures/cms_detector/tracker.pdf}\hfill
    \includegraphics[width=0.3\textwidth]{figures/cms_detector/tracking_sytem_barrel_slice.png}
    \caption[Inner Tracking System]{The left plot shows one quadrant of a
        longitudinal section of the inner tracking system consisting of the
        silicon pixel detector and the silicon strip detector. The right figure shows a
        transverse section of the tracking system in the barrel region which
        nicely illustrates the overlapping arrangement of the strip modules. Figures taken
        from~\cite{Berger:2014aca} and~\cite{cmsweb:innertracker}.}
    \label{fig:cms:inner_tracking}
\end{figure}

\subsection{Electromagnetic Calorimeter}

Measuring only the tracks of traversing particles is not sufficient to identify
the particles and derive their momentum. The energy needs to be measured as well
by stopping the particles in the detector and measuring the
deposited energy. The photon and electron energy is measured in the
electromagnetic calorimeter (ECAL). 

High-energetic photons, electrons or positrons which enter the dense material of
the ECAL detector, produce an electromagnetic shower via subsequent
bremsstrahlung and electron-pair production processes. Below a certain
threshold, the particles deposit their energy via compton scattering and the
photo-electric effect in the detector material resulting in an excitation of the
materials atomic state. Subsequently they emit photons which are measured using
avalanche photodiodes. The fraction of the deposited energy is proportional to
the number of emitted photons.

The hermetic calorimeter is made of lead tungstate ($\mathrm{PbWO}_4$), a very
dense material with a radiation length of $X_0 = \SI{0.89}{\centi\meter}$.
Through the incorporated oxygen, it is highly transparent and scintillates
light.  The small Moli\`ere  radius of \SI{2.19}{\centi\meter} leads to a fine
granularity.  These material properties allow the ECAL to be built very compact
and to be placed within the solenoid magnet. 

\todo{where is the ecal image from?}
\begin{figure}[htp]
    \centering
    \includegraphics[width=0.8\textwidth]{figures/cms_detector/cms_ecal.pdf}\hfill
    \caption[Electromagnetic Calorimeter]{The electromagnetic calorimeter
    consists of submodules covering the barrel region (EB) and the endcaps (EE).
    A complementary preshower detector (ES) is mounted in front of the
    endcaps. Taken from~\cite{Bayatian:922757}.}
    \label{fig:cms:ecal}
\end{figure}

Figure~\ref{fig:cms:ecal} shows a schematic sketch of the ECAL in the $y$-$z$
plane. The ECAL comprises three subsystems covering the the pseudo-rapidity
range up to 3.0. 

\paragraph{Electromagnetic Calorimeter Barrel (EB)} 
The EB extends up to $\eta < 1.479$ using more than \SI{60000} crystals forming a
homogenous coverage in pseudo-rapidity. Each crystal measures
$\SI{2.2}{\centi\meter} \times \SI{2.2}{\centi\meter} \times
\SI{23}{\centi\meter}$ which corresponds to 25.8 $X_0$ radiation lengths.

\paragraph{Electromagnetic Calorimeter Endcaps(EE)} 
The ECAL endcaps seal off the barrel region and extend extend the pseudo rapidity
coverage in the region $1.479 < |\eta| < 3.0$ with an additional \SI{15000}
crystals

\paragraph{Electromagnetic Pre-shower Detector (ES)} 
To incrase the spatial precision, the EE is complemented with the ES, which sits
in front of it and consists of two orthogonal silicon strip sensors. The ES
improves the discrimination between single high-energetic photos and less
interesting low-energy photon pairs as well as the discrimination between
neutral pions and photons.

The relative energy resolution of the ECAL can be parametrized using the NSC-formula

\begin{equation*}
    \left( \frac{\sigma}{E} \right)^2 = \frac{N^2}{E^2} + \frac{S^2}{E} + C^2
\end{equation*}

in which the first term describes the contribution by noise (N), the second
term the stochastic (S) component arising from the proportional relation between
the number of counted photons and the deposited energy and last a constant (C)
term.
\todo{add resolution values}

\subsection{Hadronic Calorimeter}

Hadrons entering the calorimeter produce a hadronic shower. High-energetic
hadrons mostly shower in inelastic interactions producing a large number of pions
and nucleons. Due to the large transverse momentum of the secondary particles,
hadronic shower spread further in the calorimeter than electromagnetic shower.
When the energy of the particles involved in the shower drops below a certain
theshold, the energy is deposited by ionization and low-energy hadronic
activity. The active scintillation material is excited and emits blue-violet
light. All scintillators are connected to photodiodes using wavelength
shifters which read out the signals and pass it to the data aquisition system.

The compact design of the CMS detector limits the size of the calorimeters. CMS
therefore built a sampling calorimeter inside the solenoid coil. The hadronic
calorimeter consists of brass as absorber material as it is is non-magnetic and
has a short interaction lenth of $\lambda_I = \SI{16}{\centi\metre}$. It is
interleaved with plastic scintillators measuring the deposited energy. The CMS
hadronic calorimeter comprises three subsystems. 

\paragraph{Hadron Barrel Calorimeter (HB)}
It covers the barrel region up to a pseudo-rapidity $|\eta| < 1.305$. The
absorbing material in the barrel has a corresponding thickness of
$\SI{5.39}{\lambda}$ in the central region up to $\SI{10.3}{\lambda}$ at $|\eta|
= 1.3$. The HB is complemented by the Hadron Outer Calorimeter (HO) located on
top of the coil of the magnet. Using the coil as absorbing material it is able
to meaure the tails of hadron showers penetrating the HB and the coil.

\paragraph{Hadron Endcap Calorimeter (HE)} The HE extends the pseudo-rapidity
coverage up to $|\eta| < 3.0$. A major challenge in the construction of the HE
were the usage of non-magnetic material to not disturb the magnetic field as
well as the close distance to the beampipe. Radiation damages decrease the
detector response which has to be corrected continously. 

\paragraph{Hadron Forward Calorimeter (HF)} 
The forward calorimeter extends even closer to the beam pipe. With a coverage of
$2.8 < |\eta| < 5.2$ the calorimeter is adapted to the high radiation
environment. The HF is dbuilt using iron absorbers and quartz fibres as active
material, which measure the Cerenkov light emitted by the relativistic
components of the shower.
\todo{add resolution}

\subsection{Superconducting Solenoid}

A key component of the CMS detector is the superconductiong magnet which
produces a magnetic field with a strength of \SI{4}{\tesla} and is located
inside the detector between the calorimeters and the muon system. It measures a
diameter of \SI{6}{\meter} and a length of \SI{12.5}{\meter}. When operated at
design magnetic field strength, the magnet contains an energy of
\SI{2.6}{\giga\joule}. The strong magnetic field is neccessary to bend thep
tracks of particles with high momentum to achieve a good resolution in the
tracking system. Operated at a temperature of \SI{4}{\kelvin}, the NbTi conductors become
superconducting. The magnet is complemented by a \SI{10000}{\tonne} iron yoke
which returns the magnetic field.

\subsection{Muon System}

Identifying and measuring muons with high precision is an unrivaled capacity of
the CMS detector. Unlike most other particles, muons are not stopped by the
calorimeters but leave the detector. Therefore, the muon system has been placed
around the other detector components in the iron return yoke of the magnet to
measure the bended tracks of the muons.

By combining the information of the inner tracking system and the muon
detectors, the path and the muon momentum is measured precisely. The muon
system comprises three different types of detectors each suited for a specific
task. Drift tubes (DT) cover the barrel region up to $\eta < 1.2$, the endcaps
up to $\eta < 2.4$ contain cathode strip chambers (CSC) which also
work reliably in spatially varying magnetic field. The DT and CSC detectors
yield a precise spatial muon resolution. Both system are accompanied by
resistive plate chambers (RPC) which provide fast response to the trigger
system.

\subsection{Trigger and Data Aquisition at CMS}

The LHC generates a huge number of collisions. At beam crossing frequencies of
\SI{25}{\nano \second}, there are 40 million bunch crossings per second with an
average of around 20 collisions per bunch crossing in the 2012 run period. With
todays available hardware, the storage of all collision events is not possible.
Furthermore, most of the collisions are soft and of low interest for physics
analyses. Therefore a complex trigger system consisting of a very fast
hardware-implemented component, the Level 1 trigger (L1) and a High Level
software trigger (HLT) analyze the events and accept only events which are
interesting for physics analyses.

\begin{figure}[htp]
    \centering
    \includegraphics[width=0.8\textwidth]{figures/cms_detector/cms_l1_trigger.pdf}\hfill
    \caption[The L1 Trigger of CMS]{Workflow of the L1 trigger system. The
        regional triggers search for jets and compute the transverse and missing
        energy of an event. The global calorimeter trigger sorts the objects
        from the regional calorimeter triggers and passes the top candidates to
        the Global triger, which accepts or rejects the event. If it is
        accepted, the complete data and the trigger objects are passed to the
        DAQ. Taken from~\cite{Sphicas:2002gg}.} 
    \label{fig:cms:l1_trigger}
\end{figure}

\paragraph{L1 Trigger} 
At the same frequency as collisions occur, the L1 trigger reads out the detector
electronics and analyzes the data using custom hardware. The workflow of the L1
trigger is shown in Fig.~\ref{fig:cms:l1_trigger}. Trigger Primitive Generators
calculate the transverse energy and missing energy from the frontend electronics
readout. Regional Calorimeter Triggers (RCT) identify electromagnetic showers
in the ECAL and sum up ECAL and HCAL trigger towers. Furthermore pattern
recognition is performed to identify jets and hadronic $\tau$ decays. A jet
candidate is found if the transverse energy in a region o $4\times4$ trigger
towers is greater or equal to the transverse energy of the eight surrounding
regions, see Fig.~\ref{fig:cms:l1_calo_towers}. These
candidates are passed to the Global Calorimeter Trigger (GCT) which sorts the
incoming candidates from the all 18 regional triggers and passes the top
candidates to the Global Trigger (GT). The GT accepts events with a frequency
of \SI{100}{\kilo\hertz} and passes them to the data aquisition system, which
processes the data and transfers them to the HLT.

\begin{figure}[htp]
    \centering
    \includegraphics[width=0.8\textwidth]{figures/cms_detector/l1_calo_towers.pdf}\hfill
    \caption[Jet Candidates in the Level 1 calorimeter trigger]{Jet Candidates
        in the Level 1 calorimeter trigger are formed from $4\times4$ triger
        towers. Taken from~\cite{Rose:2009zz}.}
    \label{fig:cms:l1_calo_towers}
\end{figure}
\todo{improve image}

\paragraph{HLT Trigger} 
The HLT is a software-based trigger running on a dedicated computing farm at
Point 5. The software is implemented in a streamlined version of the CMS
software framework. Each HLT trigger path is a sequence of reconstruction and
selection steps with increasing complexity. In the end, the HLT accepts several
100 events per second for permanent storage and analysis.

Jets are reconstructed in the HLT using the \antikt jet clustering algorithm. Because of the
high processing time of the Particle Flow algorithm, see
Sec.~\ref{sec:particle_flow_algorithm}, the jet trigger paths are divided into
multiple selection steps. At first, jets are reconstructed from calorimeter
towers.  Only for events in which at least one calorimeter jet passes a certain
\pt threshold, the Particle Flow algorithm is run and the jets are clustered
again from the Particle Flow candidates. Due to the flexibility of the HLT, it
is already possible to apply sophisticated jet energy corrections during the HLT
trigger selection.

\paragraph{Data Aquisition (DAQ)}

As the L1 trigger accepts events at a rate of \SI{100}{\kilo\hertz}, the DAQ
system has to process the events at the same speed. It reads out the data of all
detector subcomponents and assembles the whole events, see
Fig.~\ref{fig:cms:daq_system}. The data is subsequently passed to the HLT
trigger which further reduces the rates to a few hundred events per second.
These events are then merged and saved on a local storage system from which they
are continously transferred to the Tier 0 computing center at \CERN.

\begin{figure}[h!tp]
    \centering
    \includegraphics[width=0.8\textwidth]{figures/cms_detector/cms_daq.pdf}\hfill
    \caption[The DAQ System of CMS]{The L1 trigger accepts events at a rate of
        \SI{100}{\kilo\hertz} and passes them to the DAQ. The DAQ reads out
        all detector signals, builds the complete event and passes it to the
        HLT. All events accepted by the HLT are stored and transferred to the
        Tier 0 computing center. Taken from~\cite{Sphicas:2002gg}.}
    \label{fig:cms:daq_system}
\end{figure}

\section{Computing Infrastructure and Software Tools}

The vast amount of data produced at the LHC experiments and the complexity of
the software poses many challenges for the computing infrastructure and the
engaged software. On the theory side, powerful Monte-Carlo event generators need
to be developed which are able to simulate the collision events. On experimental
side, the physical event information needs to be reconstructed from the raw
detector readout. Furthermore the complex architecture of the detector response
needs to be modelled and simulated. These central tasks are approached using a
common software framework within CMS, the \CMSSW framework, which interfaces the
various theory tools and all the reconstruction and detector software. All of
these tasks are divided into smaller processing tasks, called jobs, which are
assigned to computing centers distributed over many countries. This common
computing and storage infrastructure is called the worldwide LHC computing grid
(LHCG).

\subsection{Worldwide LHC Computing Grid}
\todo{write}

To overcome the discussed challenges and ease the access of users to the data of
the LHC experiments, a distributed grid with a tiered infrastructure was
developed. As the majority of the data is produced at \CERN, a hierarchical
structure with the computing center at \CERN, called Tier-0 at the top was
chosen, see Fig.~\ref{fig:lhc_tier_structure}. The raw data is stored at \CERN
and distributed to globally distributed Tier-1 centers, as they provided further
storage resources and large computing resources for the reconstruction and
analysis of the data. Tier-2 sites provide additional computing ressources while
Tier-3 sites are mostly used by local analysis groups for data analyses.

\begin{figure}[htp]
    \centering
    \includegraphics[width=0.8\textwidth]{figures/cms_detector/lhcg.pdf}\hfill
    \caption[Tiered structure of the worldwide LHC Computing Grid]{The Worldwide
        LHC Computing Grid is ordered hierarchical with the \CERN T0 at the top. Taken
        from~\cite{Stober:2012abc}.}
    \label{fig:lhc_tier_structure}
\end{figure}

Access to the resources of the WLCG is gained by certificates, which authorize
the user to access the storage and computing resources.


\subsection{CMS Software Framework}

The software framework of the CMS collaboration (\CMSSW)~\cite{Bayatyan:838359}, offers
all neccessary tools for a physics analysis. The tasks in the event processing
comprise on the one hand calibration and reconstruction of data from raw
detector read-out and on the other hand the event generation and detector
simulation. Furthermore, it provides the possibility to implement analysis code
to perform the data analysis. 

To cope with this vast range of requirements to the experiment software, \CMSSW
is built on top of an event data model (EDM), in which the event is a
container for all measured or simulated data. The reconstruction and
distribution algorithms in \CMSSW are divided into modules, which can be
dynamically loaded and run. Each module reads the event data and can add additional
objects to the event. The execution of modules is ordered
in processing chains which can be configured by the user. Very often these modules
access external libraries like Monte Carlo event generators for event
simulation, Geant 4 for the detector simulation or FastJet for the
reconstruction of jets. 

While having that much information in the event data is convenient to redo
reconstruction steps, it is unsuited for fast processing in the analysis due to
its size and complexity. Therefore a skimming step, in which only the neccesary
data is preserved is run before the analysis, see Section~\ref{artus_kappa}.

\subsection{Analysis Software and Workflow}

Due to the complexity of the workflows in the HEP data analysis, several
analysis tools were used or even deloped in the Karlsruhe group to faciliate a reliable and
fast workflow of the analysis. 

\subsubsection{Artus and Kappa}
\label{artus_kappa}

The Kappa software~\cite{Kappa:2015aa} is a skimming framework interfaced to
\CMSSW. It consists of different modules which allow to skim only the physics
objects needed in the subsequent analysis. The data is stored in its distinct
compact data format using the ROOT object serialization capabilities. The
resulting Kappa tuples provide all neccesary information of the events and the
lumisections, while hiding the complexity of the \CMSSW datasets.

\begin{figure}[htp]
    \centering
    \includegraphics[width=0.8\textwidth]{figures/cms_detector/artus_workflow.pdf}
    \caption[Workflow of an analysis in the Artus framework]{Workflow of an
        analysis in the Artus analysis framework. Taken
        from~\cite{Berger:2014aca}.}
    \label{fig:artus_workflow}
\end{figure}

The analysis itself is built on top of the Artus framework~\cite{Berger:2015qao}.
Artus has been developed within the Karlsruhe group to combine analysis efforts
and to profit from mutual developments. The framework defines a workflow based
on three elements, see Fig.~\ref{fig:artus_workflow}. There are
\emph{producers}, which calculate quantities and \emph{filters}, which reject
events based on the defined criteria. In a final step, histograms or tuples are
written out by \emph{consumers}. All producers, filters and consumers are
written in a modular way so that they can be shared with other analyses.
Furthermore they are steered by a global configuration file in which all
settings and cuts can be easily adapted.

\subsubsection{grid-control}

The data sets are even after the skimming step too large to be processed on a
single computer. Therefore special batch systems with a large number of
computing nodes are employed to process the data. The data processing is split
into multiple jobs which are then sent to the batch system.
grid-control~\cite{Gridcontrol:2015aa} is by far the most versatile job
submission tool which provides multiple options for data splitting and
parametrized jobs while hiding the pitfalls of local or remote computing
resources.

\subsubsection{ROOT}

The object-oriented data analysis framework ROOT~\cite{Brun:1997pa} has been
written more than 20 years ago. However it is still very popular and used by all
LHC experiments for persistent storage of data. Moreover, ROOT provides fast
histogramming classes and access to many libraries like MINUIT for minimization
purposes or TMVA for multivariate data analyses. Despite many hours of headaches
caused by obscure design decisions in the software, ROOT and especially the
python bindings \textsc{PyROOT} are used extensively in this analysis.

\subsubsection{matplotlib and numpy}

matplotlib is a 2D plotting library written in the python programming language~\cite{Hunter:2007aa}. It
provides publication quality plots in a variety of output formats and is very
pleasant to use. All plots in this thesis were made using matplotlib. The
plotting library matplotlib as well as many scripts used in this thesis rely on
the scientific computing library NumPy~\cite{Oliphant:2007aa}. NumPy provides
powerful n-dimensional arrays and tools to manipulate them. 

\subsection{Monte-Carlo Event Generators and Simulation Software}
\label{subsection:mc_generators}

\subsubsection{Pythia}

The multi-purpose event generator Pythia simulates events in high-energy
collisons, comprising a large set of physics processes. Pythia uses the Lund
string hadronization model in which all but the highest-energy gluons are
treated as field ines which attract each other by gluon self-interaction and
form a tube or string of strong color field. In this analysis two version of the
Pythia event generator are used. The official samples including the detector
simulation were generated using Madgraph and Pythia~6~\cite{Sjostrand:2006za},
while the study of non-perturbative effects was performed using the new
Pythia~8~\cite{Sjostrand:2007gs} version, in which all the employed tunes are
available. 

\subsubsection{Herwig}

Herwig is also a multi purpose event generator for the simulation of high-energy
hadron-hadron collisions. The first version was build in Fortran and is known as
HERWIG~\cite{Corcella:2000bw}. Herwig++~\cite{Bahr:2008pv} builds up on the
heritage of the HERWIG version while providing a much more flexible structure as
it is implemented in C++. The recently released Herwig~7~\cite{Bellm:2015jjp}
version combines all their developments and supersedes both version. 

The Herwig generator family includes all steps to simulate events. It includes a
number of hard scattering processes, but also possesses the possibility to
interface external matrix element generators. The parton shower simulates
initial- and final-state radiation via angular ordering, multiple partonic
scatterings are simulated by an eikonal model and a cluster model decribes the
hadronization. 

Herwig 7 further improves these capabilities by including next-to-leading order
QCD matrix elements with matched parton showers while keeping the key features
of the previous Herwig versions. Herwig++ is used in this thesis to study
non-perturbative effects, see Sec.~\ref{sec:np_factors}. The NLO capabilities of
Herwig 7 are shown in the comparison of the unfolded measurement to NLO
predictions with matched parton showers, see Sec.~\ref{sec:nlo_comparisons}.

\subsubsection{MadGraph}

MadGraph~\cite{Alwall:2011uj} is an automated multi-purpose tree-element matrix
element generator. It implements a large number of processes and can be
interfaced to Monte Carlo Event generators. In this thesis the MadGraph matrix
elements are used together with the Pythia 6 event generator for general
comparisons to data.

\subsubsection{NLOJet++ and fastNLO}
\label{sec:nlojetpp}

The complicated NLO cross section for jet productions are calculated using
\NLOJETPP~\cite{Nagy:2003tz}. It implements the dipole subtraction method for
the separation of the divergences. \NLOJETPP can calulate up to three-jet
observables at NLO precision. It implements the ability run analysis scenarios
by which it is interfaced to \fastNLO~\cite{Kluge:2006xs,Britzger:2012bs}.

Since the pQCD cross section calculations in \NLOJETPP are determined in Monte
Carlo integration and are therefore very time consuming, it is not feasible to
repeatedly calculate the cross sections as it is neccessary for PDF fits or
uncertainty estimations. The \fastNLO framework implements a strategy for fast
recalculations of cross sections. It stores the perturbative coefficients
obtained with \NLOJETPP in a way that the strong coupling constant and the PDFs
can be changed afterwards without a recalculation of the perturbative
coefficients.

\subsubsection{LHAPDF}

All event generators and cross section calculation tools need the parton
distribution functions as input. They are either hardcoded in the generator or
accessed using a standardized interface, the LHAPDF
library~\cite{Whalley:2005nh,Buckley:2014ana}. LHAPDF stores the PDFs in a discritized
structure in data files. It provides interpolation routines to read the PDFs and
interpolate the PDFs at all scales. LHAPDF is used by almost all major MC
generators.
