% vim: tw=80

\chapter{Introduction}

Modern particle physics is driven by the desire to answer the questions about the
fundamental constituents of matter and the principles of interaction between the
constituents. High-energy collisions of particles and the analysis of the
scattering products are an optimal method to gain deep insights into the
fundamental principles of the universe. 

In hadron-hadron collisions at the LHC, point-like parton-parton scattering can
produce outgoing partons with large transverse momenta. The outgoing partons
manifest themselves as a spray of collimated particles, which are clustered
together into so-called jets. Events containing two such jets with large
transverse momenta, dijet events, allow for rigorous tests of
perturbative Quantum Chromodynamics (pQCD) predictions and can subsequently be
used to also gain a better understanding of the inner structure of the proton
and to determine the strong coupling constant.

Especially in view of upcoming NNLO corrections for dijet calculations in
perturbative QCD, dijet observables are an optimal candidate for precision
studies of the proton. Consequently, such events were studied and an
optimal observable was found in the triple-differential dijet cross section.

This thesis presents the theoretical motivation and the measurement of the
triple-differential dijet cross section, which is measured as a function of the
average transverse momentum of the two leading jets and binned in the rapidity
separation and the boost of the dijet system. This measurement optimally relates
the dijet cross section originating from different dijet topologies to the
underlying momentum fractions of the proton which is demonstrated thereupon by a
combined fit of the PDFs to deep-inelastic scattering data from the HERA
experiments and the triple-differential dijet cross section.

\begin{figure}[h!tb]
    \centering
    \includegraphics[width=0.4\textwidth]{figures/drawings/ybys_hint.pdf}
\end{figure}

In Chapter~\ref{sec:theoretical_foundations}, the theoretical foundations for
jet production at hadron-hadron colliders are outlined. An overview of the
Standard Model of particle physics with a focus on perturbative quantum
chromodynamics is given. Chapter~\ref{sec:experimental_setup} summarizes the
experimental setup of the LHC collider and the CMS detector. Furthermore, the
employed software tools and Monte Carlo event generators are described. The
reconstruction of jets is detailed in Chapter~\ref{sec:jet_reconstruction}.

The theoretical motivation and the definition of the dijet observables are
introduced in Chapter~\ref{sec:theory_predictions}, in which also the accuracy
of the NLO pQCD calculation is studied. The measurement of the
triple-differential dijet cross section is explained in
Chapter~\ref{sec:measurement}. The cross section is determined from the data
collected by CMS at \SI{8}{\TeV} which corresponds to an integrated luminosity
of \SI{19.71}{\per \femto \barn}. The measurement is corrected for detector
effects in an unfolding procedure and compared to NLO predictions calculated in
pQCD.

Finally, the sensititivity of the proton PDFs to the measured data is
demonstrated in Chapter~\ref{sec:pdf_constraints} resulting in constraints on
the PDFs, especially the gluon PDF. Moreover, a simultaneous fit of the PDFs and
the strong coupling constant is presented.

% \layout
