% vim: tw=80

\chapter{Introduction}


% jet production at hadron colliders
% knowledge pdfs improve

% triple differential dijet cmeasure

% zusammenfassungen jedes kapitel

The present thesis presents a dijet cross section measured, which is measured as
function of the average transverse momentum of the two leading jets and binned
in the rapidity separation and the boost of the dijet system. The measurement is
well suited to study the structure of the proton and to improve the current
knowledge on the parton distribution functions (PDFs). 

In Chapter~\ref{sec:theoretical_foundations}, the theoretical foundations for
jet production at hadron-hadron colliders are outlined. An overview of the
Standard Model of particle physics with a focus on perturbative quantum
chromodynamics is given. Chapter~\ref{sec:cms_detector} summarizes the
experimental setup of the LHC collider and the CMS detector. Furthermore, the
employed software tools and Monte Carlo event generators are described. The
reconstruction of jets is detailed in Chapter~\ref{sec:jet_reconstruction}.

The theoretical motivation and the definition of the dijet observables are
introduced in Chapter~\ref{sec:theory_predictions}, in which also the accuracy
of the NLO pQCD calculation is studied. 

The measurement of the triple-differential dijet cross section is explained in
Chapter~\ref{sec:measurement}. The cross section is determined from the data
collected by CMS at \SI{8}{\TeV} which corresponds to an integrated luminosity
of \SI{19.71}{\per \femto \barn}. The measurement is unfolded and compared to
NLO predictions calculated in pQCD.

The sensititivity of the proton PDFs to the measured data is studied in
Chapter~\ref{sec:pdf_constraints} resulting in constraints of the PDFs,
especially the gluon PDF.
