% vim: tw=80

\chapter{Theory Predictions for the Triple-Differential Dijet Cross Section}
\label{sec:theory_predictions}

Point-like parton-parton scattering processes in high-energy collisions can
produce jets with large transverse momenta. Events containing two such jets in
the final state, also known as dijet events, allow for rigorous tests of
perturbative Quantum Chromodynamics predictions and can subsequently be used to
further constrain the proton PDFs and extract Standard Model parameters like the
strong coupling constant~\as.

Next-to-leading-order (NLO) pQCD predictions have been available for dijet and
multijet observables since many years. These accurately describe shape and
normalization of jet cross sections, although they still suffer from larger
scale uncertainties limiting the precision of Standard Model parameters
extracted from measurements.

For almost ten years, theorists have been working on improving the jet cross
section predictions by providing next-to-next-to-leading-order (NNLO)
corrections for dijet calculations. When these corrections become publicly
available, they will push the precision of pQCD jet cross section predictions to
a new level.

With the finalization of this huge project steadily approaching, we provide a
measurement which is specifically designed to benefit from these enhancements.
The ultimate aim of this measurement is the improvement of the proton PDFs,
especially the gluon PDF, by analyzing dijet events with large transverse
momenta.

Common cross section measurements like the inclusive jet cross section are well
understood but have the disadvantage that PDF sensitive and insensitive phase
space regions are not well separated which limit the constraining power of these
measurements.

Therefore, triple-differential cross section observables were studied in this
thesis, which promise to yield the most information about the proton structure.

Basically, it is possible to measure the energy of the dijets in bins of the
rapidities of the leading and second jet. However by explicitly binning the
measurement in the leading and second jet as it is suggested
in~\cite{Giele:1994xd}, a dependence on the ordering of the jets is introduced.
While the ordering according to the transverse momentum of the jets is
irrelevant at LO since both jets are perfectly balanced and have the same
transverse momentum, it becomes relevant at NLO. The order of the jets can be
changed by a soft emission causing a different ordering of the jets in \pt.
Simply put, the second jet can become the leading jet and vice versa. Thus it is
not guaranteed that all divergences are canceled. As a consequence the
observable becomes infrared unsafe. This can be overcome by filling all events
twice into the histograms with interchanged leading and second jet. However this
produces correlations between phasespace regions far off in rapidity and
unnecessarily complicates the measurement, especially the unfolding procedure.

Therefore a different definition of the cross section is chosen in this
analysis. Instead of using the rapidities and the transverse momenta of each
jet, variables which are symmetric between permutations of the leading two
jets are used. As these variables are linear combinations of the jet rapidities,
no information is lost. Sec.~\ref{sec:crosssection_definition} presents the
cross section definition chosen for the analysis. The rest of chapter presents
NLO pQCD predictions for the cross section as well as an detailed study of all
sources of theoretical uncertainties afflicting the cross section calculations.

\section{Definition of the Cross Section}
\label{sec:crosssection_definition}

The triple-differential dijet cross section is motivated by its relation of the
kinematic properties of the dijet system to the fractional proton momenta. An
introduction to the kinematics of the dijet system is given
in~\ref{sec:dijet_kinematics}. The triple-differential dijet cross section,

\begin{equation*}
    \frac{d \sigma}{d \ptavg d\ystar d \yboost}
\end{equation*}

is measured as a function of average transverse momentum \ptavg and binned in
half the absolute rapidity separation \ystar and the boost of the dijet system
\yboost. The variables are defined as

\begin{align*}
\ptavg &=\frac{1}{2}(p_{\mathrm{T},1} + p_{\mathrm{T},2})\\
\yboost& =\frac{1}{2}|y_1 + y_2|\\
\ystar &=\frac{1}{2}|y_1 - y_2|
\end{align*}

where $p_{\mathrm{T},1}$ and $p_{\mathrm{T},2}$ denote the transverse momentum,
and $y_1$ and $y_2$ the rapidities of the leading and second leading jet.
Kinematic cuts of the two leading jets ensure the comparability of the
measurement and the calculations:

\begin{align*}
    \ptjet &> \SI{50}{\GeV}\\
    |y_\mathrm{jet}| &\leq 3.0\\
    \ptavg &> \SI{133}{\GeV}
\end{align*}

The phase space cuts are motivated by the acceptance of the detector. The cut on
\ptavg results from the jet triggers employed in the measurement which reach
full efficiency at \SI{133}{\GeV}.

\begin{figure}[h!tbp]
    \centering
    \includegraphics[width=0.9\textwidth]{figures/measurement/ybys.pdf}
    \caption[Dijet topologies in \ystar and \yboost phasespace]
        {A descriptive representation of the dijet topologies in the various
    \ystar and \yboost bins. The binning separates in same side dijet
    events and opposite side dijet events which allows to draw conclusions about
    the initial state parton properties.}
    \label{fig:ysyb_schema}
\end{figure}

The binning of the cross section in \ystar and \yboost has the additional
advantage same-side (SS) and opposite-side (OS) dijet events are separated into
different bins. Fig.~\ref{fig:ysyb_schema} depicts the dijet topologies in the
various \ystar and \yboost bins. If the rapidity separation and the boost of the
dijet event are small, both jets must have a low rapidity and consequently also
small \ystar and \yboost values. These events are filled in the bottom left bin.
If the dijet system is boosted but the two jets have a small separation in
rapity, both jets must be boosted in the forward region to the same side, see
bottom right . If instead the dijet separation is large and the boost of the
dijet system is small, the jets are boosted to opposide sides. Dijet events with
both a large rapidity separation and a large boost of the dijet system are
suppressed in the accessible phase space.

The differentiation in SS and OS dijet events is especially interesting since
both event topologies must access different fractional proton momenta while the
jets manifest themselves in the same (forward) detector region. Therefore,
differences in the predictions can be attributed to the PDFs. This is clearly
visible in Fig.~\ref{fig:pdf_uncertainties}, which shows the PDF uncertainties of
the cross section calculations. The measurement bin containing the events with
the large rapidity separation, see bottom right plot, has a significantly larger
PDF uncertainty since the high fractional proton momenta are accessed and the
PDFs are less precisely known in this region.

\section{Fixed Order NLO Prediction}

The NLO cross section of the triple-differential dijet cross section is
calculated using fastNLO~\cite{Kluge:2006xs,Britzger:2012bs} which uses
interpolation tables filled with the perturbative coefficients of the \NLOJETPP
program~\cite{Nagy:2003tz}, see Sec.~\ref{sec:nlojetpp}. The PDFs are accessed
via the LHAPDF library~\cite{Whalley:2005nh,Buckley:2014ana} and the
\as-evolution is performed using the routines provided by the PDF sets.
Employing \fastNLO instead of a direct calculation with \NLOJETPP gives the
possibility to repeatedly calculate the cross section with  different PDFs and
scale choices as it is necessary for the calculation of PDF and scale
uncertainties.

\subsection{Scale Choice}
\label{sec:scale_coice}

Within perturbative cross section calculations, one has to choose a
factorization scale \muf and a renormalization scale \mur. The influence of
these scales vanishes if the calculation is performed for all orders of the
perturbative series. However, since the perturbative series is truncated at
NLO, a scale dependence on the result remains. Three possibilities for the scale
choice are studied in this thesis.  The most natural scale choice is the average
transverse momentum  of the dijet system which is also used as observable and
therefore reflects the energy scale of the measurement.

\begin{equation*}
    \mu = \mur = \muf = \frac{p_{\mathrm{T},1} + p_{\mathrm{T},2}}{2}
\end{equation*}

While this scale choice yields reasonable results, the $k$-factors and scale
uncertainties indicate problems, which are discussed in detail in
Sec.~\ref{sec:k_factors} and Sec.~\ref{sec:scale_uncertainties}. The second
investigated scale choice is based on the findings of a recent analysis by the
ATLAS collaboration~\cite{Aad:2011fc}. They claim that fixed-order calculations
which are binned in the rapidity separation \ystar become  unreliable for high
values of \ystar if the scale choice only depends on the energy. Based on
recommendations of theorists~\cite{Ellis:1992en}, a scale which also depends on
the rapidity separation is proposed.
\todo{more motivation}

\begin{equation*}
    \mu = \mur = \muf = p_{\mathrm{T,max}} e^{0.3 \ystar} 
\end{equation*}

Furthermore a variation of this scale choice is studied in which the scale is not
dependent on the transverse momentum of the leading jet but on the average
transverse momentum of the leading two jets as this again resembles the
observable in the measurement.

\begin{equation*}
    \mu = \mur = \muf = p_{\mathrm{T,avg}} e^{0.3 \ystar} 
\end{equation*}

Fig.~\ref{fig:xs_nlo_comp} shows the predictions of the NLO calculation using
the three discussed scale choices. The cross sections predicted by each
calculation are similar with somewhat larger deviations for the scale choice using the
maximum transverse momentum instead of the average dijet transverse momentum.
The differences between the predictions, however, are covered by the scale
uncertainties.

\begin{figure}[htp]
    \centering
    \includegraphics[width=0.45\textwidth]{figures/theory/nlo_xs_comp_yb0ys0.pdf}\hfill
    \includegraphics[width=0.45\textwidth]{figures/theory/nlo_xs_comp_yb0ys1.pdf}
    \includegraphics[width=0.45\textwidth]{figures/theory/nlo_xs_comp_yb0ys2.pdf}\hfill
    \includegraphics[width=0.45\textwidth]{figures/theory/nlo_xs_comp_yb1ys0.pdf}
    \includegraphics[width=0.45\textwidth]{figures/theory/nlo_xs_comp_yb1ys1.pdf}\hfill
    \includegraphics[width=0.45\textwidth]{figures/theory/nlo_xs_comp_yb2ys0.pdf}
    \caption[fastNLO prediction of triple-differential dijet cross section]{
        NLO predictions of \fastNLO interfaced to \NLOJETPP for the
        triple-differential dijet measurement. The calculations using three
        different scale choices are shown revealing differences up to 10\%. In
        almost all cases the differences between the calculations are covered by
        the scale uncertainties, see Sec.~\ref{sec:scale_uncertainties}.}
    \label{fig:xs_nlo_comp}
\end{figure}


\subsection{NLO Correction Factors}
\label{sec:k_factors}

To check the influence of higher-order contributions to the perturbative QCD
prediction, the differences between the LO prediction and the NLO
prediction are studied, here expressed as the ratio $k_\mathrm{NLO}$.

\begin{equation*}
    k_{\mathrm{NLO}} = \frac{\sigma_{\mathrm{NLO}}}{\sigma_{\mathrm{LO}}}
\end{equation*}

The size of the NLO correction gives an estimation about the influence of these
higher-order corrections. If they are small, the LO result already describes the
observable cross section precisely. It is also possible that the $k$-factors
fall below unity, in which case the NLO corrections are negative and the total cross
section decreases when adding the correction.  Fig.~\ref{fig:kfactor_comp} shows
the $k$-factors of the \NLOJETPP cross section calculations using the discussed
scale choices in Sec.~\ref{sec:scale_coice}. The $k$-factors are similar in the
central region, but the differences increase in regions with larger rapidity
separations. Especially the $k$-factors in the phase space region with a
rapidity separation of $2.0 \leq \ystar < 3.0$ are smaller than unity for the
\ptavg scale choice, while it is larger than one for the scale choices including
the \ystar dependence.

Apart from the findings for the \ptavg scale choice, the $k$-factors are
reliable and meet the expectations from previous studies of jet observables.

\begin{figure}[htp]
    \centering
    \includegraphics[width=0.45\textwidth]{figures/theory/kfactor_comp_yb0ys0.pdf}\hfill
    \includegraphics[width=0.45\textwidth]{figures/theory/kfactor_comp_yb0ys1.pdf}
    \includegraphics[width=0.45\textwidth]{figures/theory/kfactor_comp_yb0ys2.pdf}\hfill
    \includegraphics[width=0.45\textwidth]{figures/theory/kfactor_comp_yb1ys0.pdf}
    \includegraphics[width=0.45\textwidth]{figures/theory/kfactor_comp_yb1ys1.pdf}\hfill
    \includegraphics[width=0.45\textwidth]{figures/theory/kfactor_comp_yb2ys0.pdf}
    \caption[NLO $k$-factors of fastNLO calculation]{The $k$-factors between the NLO calculation and LO calculation
        shows the influence of the NLO correction terms. The $k$-factors for the
        calculation with \ptavg as scale choice fall below unity in case of high
        \ystar values indicating that the NLO correction is negative in this
        phase space region.}
    \label{fig:kfactor_comp}
\end{figure}

\section{NLO Prediction with Matched Parton Showers}

Cross section calculations beyond leading order combining parton showers, MPI
and hadronization models are very complicated. Double counting of terms in the
perturbative series and the parton shower needs to be avoided. The recently
released version 7 of Herwig includes NLO predictions with matched parton
showers.  The calculation uses the MC@NLO type matching as it is implemented in
the Matchbox module~\cite{Platzer:2011bc} of Herwig. The PDFs of the MMHT 2014
NLO PDF set are accessed using LHAPDF. For the calculation of the matrix
elements, the NJet library~\cite{Badger:2012pg} is interfaced to Herwig and the
renormalization and factorizaton scale are set to $\mu = \ptmax$. As the
integration and event generation steps are extremely time consuming, the phase
space was split into 12 mutually exclusive regions binned in the transverse
momentum of the leading jet and the phase space regions were stitched together
on analysis level.

\todo{add and describe figure with results.}

\section{Non-Perturbative Corrections}
\label{sec:np_factors}
\todo{nlo np factors hw7}

The perturbative QCD calculations of \NLOJETPP give a cross section at NLO
parton level. These so-called fixed-order pQCD calculations cannot be directly
compared to the unfolded data, as they do not include additional soft QCD
effects. While a part of these effects is absorbed by the jet-algorithm,
they still must be estimated and accounted for in comparisons of fixed-order
calculations to data.

The influence of these soft effects are estimated using Monte Carlo event
generators which are able to simulate those. The usual approach consists of
calculating the cross section including effects from multi-parton interactions
(MPI) and hadronization. The non-perturbative (NP) correction $c_k^\mathrm{NP}$
obtained with a MC event generator $k$ is defined as the ratio between the
nominal cross section including MPI and hadronization effects and a cross section calculation
neglecting those, see Eq.~\ref{eq:np_definition}. The superscript in the
equation indicates the applied steps in the simulation: the parton shower (PS),
the multi-parton interaction model (MPI) and the hadronization (HAD). The
correction is then applied as a bin-by-bin correction factor to the NLO cross
section.

\begin{equation*}
    c_{k}^{\mathrm{NP}} = \frac{\sigma^{\mathrm{PS+HAD+MPI}}}{\sigma^{\mathrm{PS}}}
    \label{eq:np_definition}
\end{equation*}

The NP corrections have been calculated by employing two Monte Carlo
event generators using the newest available tunes within CMS. Herwig++ is used
with the tune UE-EE-5C and Pythia 8 with the tune CUETP8M1. The ratio is fitted
by a power-law function
\todo{ref to tune}

\begin{equation*}
  f(\ptavg) = a \cdot \ptavg^b + c.
  \label{fcn:np_fit}
\end{equation*}

Since the correction factors obtained from Herwig++ and Pythia8 are quite
different in the low-\pt region, an uncertainty was assigned to the correction
factor. The correction factors are determined by the average of the factors
obtained from the Pythia 8 and Herwig++ predictions.

\begin{equation*}
    c^\mathrm{NP} = \frac{c_{\mathrm{P8}}^{\mathrm{NP}} + c_{\mathrm{HW++}}^{\mathrm{NP}}}{2}
\end{equation*}

The uncertainty quoted for the NP correction factors is half the difference
between the two correction factors.

Fig.~\ref{fig:np_factors} shows the resulting correction factors and the
corresponding uncertainty. The corrections vary between 10\% and 15\% at
\SI{100}{\GeV} and get smaller at higher values of \ptavg. While the correction
decreases for higher transverse momenta, it does not approach unity, especially
in the bins containing boosted jets. 

\begin{figure}[htp]
    \centering
    \includegraphics[width=0.45\textwidth]{figures/theory/np_factors_calc_yb0ys0.pdf}\hfill
    \includegraphics[width=0.45\textwidth]{figures/theory/np_factors_calc_yb0ys1.pdf}
    \includegraphics[width=0.45\textwidth]{figures/theory/np_factors_calc_yb0ys2.pdf}\hfill
    \includegraphics[width=0.45\textwidth]{figures/theory/np_factors_calc_yb1ys0.pdf}
    \includegraphics[width=0.45\textwidth]{figures/theory/np_factors_calc_yb1ys1.pdf}\hfill
    \includegraphics[width=0.45\textwidth]{figures/theory/np_factors_calc_yb2ys0.pdf}
    \caption[Non-perturbative Corrections]{The non-perturbative corrections are derived using a MC event
        generator by calculating the ratio of the cross section with and without
        hadronization and MPI effects enabled in the simulation. The green line
        shows the correction of Pythia 8, the blue line the correction of
    Herwig++. The average of both gives the central correction and the envelope
represents the uncertainty on the correction.}
    \label{fig:np_factors}
\end{figure}

\section{Theory Uncertainties}

Multiple sources of uncertainty limit the precision of the NLO cross section
calculation. In this section, the derivation of the scale and PDF
uncertainties is described and a comparison of all studied theoretical
uncertainties is shown in Fig.~\ref{fig:theo_uncertainties}. The NP
uncertainties are discussed in Sec.~\ref{sec:np_factors}, in which the
non-perturbative corrections are determined. 

The scale uncertainty is the dominant source of uncertainty in the low-\pt
region and is of the size of 5\% to 10\% in most of the phase space regions. For increasing
transverse momenta of the dijet system, the PDF uncertainty becomes the
dominant source. Its size ranges from 5\% in the low-\pt region up to
50\% for highest \pt. The uncertainty afflicted to non-perturbative corrections
is only sizeable for lower \pt and is less than 5\% in all bins.

As the cross section calculation is performed using \fastNLO interfaced to
\NLOJETPP, multiple independent calculations can be merged to increase the
statistical precision. The statistical uncertainty is estimated by calculating
the uncertainty on the arithmetic mean of the cross section of all fastNLO
tables. It is found to be smaller than 0.5\% in all
bins, in most of the bins even smaller than 0.1\%. Therefore, the statistical
uncertainty is neglected in all further comparisons.

\subsection{Scale uncertainties}
\label{sec:scale_uncertainties}

As discussed in Sec.~\ref{sec:scale_coice}, one has to choose a factorization
and renormalization scale for a perturbative cross section calculation. Due to
the truncated perturbative series, a scale dependence remains. The effects of
neglected higher-order contributions are covered by a systematic theoretical
uncertainty.

There is a common approach in estimating the influence of the scale on the cross
section~\cite{Cacciari:2003fi}. The cross section calculation is performed using
multiple scale choices and the differences compared to the cross section
obtained using the central scale choice are translated into a scale uncertainty.
The variations are applied as multiplicative factors to the central choice in
the following six combinations: $(\mur, \muf) = (\sfrac{1}{2}, \sfrac{1}{2}),
(\sfrac{1}{2}, 1), (1, \sfrac{1}{2}), (1, 2), (2, 1)$ and $(2, 2)$ times new
nominal scale. The uncertainty on a quantity $X$, \eg the cross section, is the
envelope of the maximum deviation in the upwards and downwards direction, while
$X^0$ denotes the value of the quantity with the default scale choice and $n$
denotes the number of variations.

\begin{align*}
    \Delta X^+ &= \max_{i}^{n} \left[ X^i - X^0, 0 \right]\\
    \Delta X^- &= \max_{i}^{n} \left[ X^0 - X^i, 0 \right]
\end{align*}


Fig.~\ref{fig:scale_uncertainties} shows the relative size of the scale
uncertainty for each bin of the measurement and the discussed scale choices. The
scale uncertainty if of the size of 5\% to 10\%. However, when using the scale
choice $\mu=\ptavg$, the bin with the largest rapidity separation exhibits a large
increase of the uncertainty of up to 40\%, which is undesired.

\begin{figure}[htp]
    \centering
    \includegraphics[width=0.45\textwidth]{figures/theory/scale_uncert_comp_yb0ys0.pdf}\hfill
    \includegraphics[width=0.45\textwidth]{figures/theory/scale_uncert_comp_yb0ys1.pdf}
    \includegraphics[width=0.45\textwidth]{figures/theory/scale_uncert_comp_yb0ys2.pdf}\hfill
    \includegraphics[width=0.45\textwidth]{figures/theory/scale_uncert_comp_yb1ys0.pdf}
    \includegraphics[width=0.45\textwidth]{figures/theory/scale_uncert_comp_yb1ys1.pdf}\hfill
    \includegraphics[width=0.45\textwidth]{figures/theory/scale_uncert_comp_yb2ys0.pdf}
    \caption[Scale Uncertainties of NLO calculation]{The scale uncertainty of
        the cross section calculation in each of the six bins in \ystar and
        \yboost. The uncertainty is estimated using the common approach of
        independently varying the renormalization and factorization scale choice
        in six independent combinations. The uncertainty is shown for the three
        investigated scale choices, indicated by different colors. In most cases
        the scale uncertainty is reasonable and of the size of 5\% to 10\%. In
        the region with the largest values of \ystar, the scale uncertainty of
        the prediction with the scale choice $\mu=\ptavg$ is undesirably large.}
    \label{fig:scale_uncertainties}
\end{figure}

\subsection{PDF uncertainties}
\label{sec:pdf_uncertainties}

The dependence of the cross section calculation on the proton structure is
expressed in terms of parton distribution functions, which are derived from
fits to data from several experiments. Different sources of
uncertainty affect the PDFs. These comprise the choice and the
functional form of the parametrization, the chosen theory model and input
parameters like the strong coupling constant \as or the quark masses and, of
course also the statistical and systematic uncertainty sources of the data points included in the PDF fit.

When determining the PDFs, all these uncertainties are taken into account and
are propagated in the PDFs. The groups deriving the PDF sets provide
prescriptions how to evaluate these uncertainties. In
Sec.~\ref{sec:nlo_comparisons}, several comparisons of unfolded data to predictions using the
global PDF sets NNPDF 3.0, CT14 and MMHT2014 are shown. In the following a short
summary of the procedure to derive the PDF uncertainties for these PDF sets is
given.

The NNPDF PDF set~\cite{Ball:2014uwa} uses a large number of pseudo experiments, in which the PDF
fit is performed using data smeared within their uncertainties while taking into
account all correlations. These so-called replicas are averaged to give the
central result $X_\mathrm{central}$ and the spread of the replicas determines the uncertainty. The
symmetric PDF uncertainty $\Delta X^\pm$ of a quantity $X$, which can be a cross section calculation or
even the PDF itself is expressed as

\begin{equation*}
    \Delta X^{\pm} = \sqrt{\frac{1}{N-1} \sum_{i=1}^N \left[ X_{i} - X_{\mathrm{central}} \right]^2}
\end{equation*}
where $N$ denotes the number of replicas.

The CT14 PDF set ~\cite{Dulat:2015mca} and the MMHT2014 PDF
set~\cite{Harland-Lang:2014zoa} both employ the eigenvector method to encode the
uncertainties. A transformation from the parameter basis to the eigenvector
basis is done to yield mutually uncorrelated eigenvectors. By varying the
eigenvectors upwards and downwards, a set of eigenvector pairs is generated
which can be used to determine the asymmetric uncertainty $\Delta X^+$ and
$\Delta X^-$ of a quantity $X$. $X_0$ denotes the central prediction,
$X_i^{\mathrm{up}}$ and $X_i^{\mathrm{dn}}$ are the predictions using the
upwards and downwards variation of the eigenvector PDF set $i$ and
$N_{\mathrm{EV}}$ is the number of eigenvectors in the PDF set.

\begin{equation*}
\begin{aligned}
    \Delta X^+ &= \sqrt{\sum_i^{N_{\mathrm{EV}}} \left[ \max(X_i^{\mathrm{up}}
    -X_0, X_i^{\mathrm{dn}} - X_0, 0)\right]^2}\\
\Delta X^- &= \sqrt{\sum_i^{N_{\mathrm{EV}}} \left[ \min(X_i^{\mathrm{up}} - X_0, X_i^{\mathrm{dn}} - X_0,0)\right]^2}
\end{aligned}
\end{equation*}

The symmetric uncertainty $\Delta X^{\pm}$ is given by half the difference of the upwards and
downwards variation.

\begin{equation*}
    \Delta X^{\pm} = \sqrt{\sum_i^{N_{\mathrm{EV}}} \left[ \frac{X_i^+ -
    X_i^-}{2} \right]^2}
\end{equation*}

The uncertainty assigned to the CT14 PDF set describes a 90\% confidence
interval (CI), while the MMHT and NNPDF PDF uncertainties represent a 68\% CI.
The CT14 uncertainties are scaled to 68\% CI using $s = \sqrt{2}
\erf^{-1}(0.9) = 1.645$.

Fig.~\ref{fig:pdf_uncertainties} shows the fractional PDF uncertainty for the
three global PDF sets discussed. The PDF uncertainty in the bins with small
\ystar and small \yboost values is comparatively small. This is due to the fact
that mostly events with opposite side jets contribute, in which the medium $x$ region
of the proton PDFs is accessed which are well known already. The PDF uncertainty
of the cross section for high values of \ystar and low values of \yboost in
which two forward jets are on opposite sides of the detector is also relatively
small. Most interestingly the uncertainty strongly increases for the bin with
largest values of \yboost and small values of \ystar.  Especially in the high-\pt
region, the uncertainty is sizable. To achieve a high boost of the dijet system
and a high \ptavg value, the protons must be accessed in the high-$x$ region
which is not well determined up to now and is afflicted with large PDF
uncertainties. Especially the NNPDF PDF set has a large uncertainty in this
region. This is caused by the very flexible parametrization of the NNPDF PDF set
which results in large PDF uncertainty in phase space regions not covered by
data.

\begin{figure}[htp]
    \centering
    \includegraphics[width=0.45\textwidth]{figures/theory/pdf_unc_comparison_yb0ys0.pdf}\hfill
    \includegraphics[width=0.45\textwidth]{figures/theory/pdf_unc_comparison_yb0ys1.pdf}
    \includegraphics[width=0.45\textwidth]{figures/theory/pdf_unc_comparison_yb0ys2.pdf}\hfill
    \includegraphics[width=0.45\textwidth]{figures/theory/pdf_unc_comparison_yb1ys0.pdf}
    \includegraphics[width=0.45\textwidth]{figures/theory/pdf_unc_comparison_yb1ys1.pdf}\hfill
    \includegraphics[width=0.45\textwidth]{figures/theory/pdf_unc_comparison_yb2ys0.pdf}
    \caption[PDF uncertainty]{The relative PDF uncertainty is calculated using the three PDF sets
    NNDFP 3.0, CT14 and MMHT 2014. The uncertainty represents a 68\% confidence
    interval. The PDF uncertainty is sizable especially in the forward region in
    which both jets have the same sign and high fractional proton momenta are
    accessed. }
    \label{fig:pdf_uncertainties}
\end{figure}

\begin{figure}[htp]
    \centering
    \includegraphics[width=0.45\textwidth]{figures/theory/theo_unc_yb0ys0.pdf}\hfill
    \includegraphics[width=0.45\textwidth]{figures/theory/theo_unc_yb0ys1.pdf}
    \includegraphics[width=0.45\textwidth]{figures/theory/theo_unc_yb0ys2.pdf}\hfill
    \includegraphics[width=0.45\textwidth]{figures/theory/theo_unc_yb1ys0.pdf}
    \includegraphics[width=0.45\textwidth]{figures/theory/theo_unc_yb1ys1.pdf}\hfill
    \includegraphics[width=0.45\textwidth]{figures/theory/theo_unc_yb2ys0.pdf}
    \caption[Overview of theoretical uncertaintites]{Overview of the theoretical uncertainties of the NLO prediction.
    The scale uncertainty is the dominant uncertainty in the low-\pt region. At
    high-\pt and especially in the forward region, the PDF uncertainty becomes
    dominant. The NP uncertainty is sizable only in the low-\pt region and becomes
    negligible at higher \pt when the NP corrections approach unity.}
    \label{fig:theo_uncertainties}
\end{figure}
