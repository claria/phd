\chapter{PDF Constraints}

% The PDFs of the proton are an essential ingredient in precision
% studies. PDFs are derived from experimental data involving collider
% and fixed-target experiments. Deep-inelastic scattering (DIS) data
% delivered by the HERA collider cover most of the kinematic phase space
% needed for a reliable PDF extraction. The $pp$ inclusive jet cross
% section contains additional information and can constrain the PDFs, in
% particular for the gluon, in the region of high fractions $x$ of the
% proton momentum.
%
% The HERAFitter project~\cite{Aaron:2009kv,Aaron:2009aa,HERAFitter:2013hf} is an open
% source fitting framework designed to derive PDFs from data. It has a
% modular structure, encompassing a variety of theoretical predictions
% for different processes and phenomenological approaches for the PDF
% determination. Confronting data with theoretical predictions the
% parton distributions are determined.  In this study HERAFitter is
% employed to estimate the impact of the CMS inclusive jet data on the
% PDFs and their uncertainties using the fixed-order theory predictions
% and NP corrections as defined in the previous section.



\section{Sensitivity of PDFs to Dijet Measurement}
\label{sec:pdf_sensitivity}

% The potential impact of the CMS inclusive jet data can be illustrated
% by the correlation between the inclusive jet cross section
% $\sigma_{\mathrm{jet}}(x,Q^2)$ and the PDF $xf_i(x,Q^2)$ for each
% parton of flavour $i$. As before, $x$ is the fraction of the proton
% momentum, and $Q$ is the relevant energy scale of the hard process,
% which in the case of inclusive jets is identified with the jet \pt.
%
% The NNPDF Collaboration~\cite{Ball:2008by} provides PDF sets in the
% form of an ensemble of PDFs called replicas, which sample variations
% in the PDF parameter space allowed within uncertainties. Evaluating
% means and variance with the help of these replicas, predictions
% including uncertainties are derived for PDF dependent observables.
% For an NNPDF ensemble with $N_{\mathrm{rep}}$ replicas, the
% correlation coefficient $\varrho$ between a cross section and the PDF
% for flavour $i$ is then defined as:
%
% \begin{equation}
%   \varrho_i \left[ \sigma_{\mathrm{jet}}(x,Q^2) , xf_i(x,Q^2) \right] =
%   \frac{N_{\mathrm{rep}}}{(N_{\mathrm{rep}}-1)} \frac{\langle
%     \sigma_{\mathrm{jet}}(x,Q^2) xf_i(x,Q^2) \rangle -
%     \langle \sigma_{\mathrm{jet}} \rangle \langle xf_i(x,Q^2)
%     \rangle}{\Delta_{\sigma_{\mathrm{jet}}(x,Q^2)} \Delta_{xf_i(x,Q^2)}}\,.
% \end{equation}
%
% Here, the uncertainties of the jet cross sections and the PDFs
% themselves are represented by $\Delta_{\sigma_\mathrm{jet}}$ and
% $\Delta_{xf(x,Q^2)}$, respectively.
% Figure~\ref{fig:correlation_pdf_xs_gqq} presents the correlation
% coefficient between the inclusive jet cross section and the gluon, up,
% and down quark PDFs in the proton.
%
% \begin{figure}[h!tb]
%   \centering
%   \includegraphics[width=0.45\textwidth]{herafitter_Figures/correlation_plots/corr_fnl2332d_NNPDF21_gluon_y0_0_bw}\hfill%
%   \includegraphics[width=0.45\textwidth]{herafitter_Figures/correlation_plots/corr_fnl2332d_NNPDF21_gluon_y2_0_bw}
%   \includegraphics[width=0.45\textwidth]{herafitter_Figures/correlation_plots/corr_fnl2332d_NNPDF21_u_quark_y0_0_bw}\hfill%
%   \includegraphics[width=0.45\textwidth]{herafitter_Figures/correlation_plots/corr_fnl2332d_NNPDF21_u_quark_y2_0_bw}
%   \includegraphics[width=0.45\textwidth]{herafitter_Figures/correlation_plots/corr_fnl2332d_NNPDF21_d_quark_y0_0_bw}\hfill%
%   \includegraphics[width=0.45\textwidth]{herafitter_Figures/correlation_plots/corr_fnl2332d_NNPDF21_d_quark_y2_0_bw}
%   \caption{The correlation coefficient between the inclusive jet cross
%     section and the gluon (top row), the up quark (middle row), and
%     the down quark PDF (bottom row) as a function of the momentum
%     fraction $x$ of the proton and the momentum scale $Q$ of the hard
%     process. The correlation is shown for the central rapidity region
%     $|y|<0.5$ (left) and for $2.0<|y|<2.5$ (right).}
%   \label{fig:correlation_pdf_xs_gqq}
% \end{figure}
%
% The correlation between the gluon PDF and the inclusive jet cross
% section is observed to be largest at central rapidity for all jet
% transverse momenta. In contrast the correlation between the quark
% distributions and the jet cross section is rather small in this
% kinematic region such that a large impact on the quark distribution is
% not expected from including these jet data into PDF fits. In the
% forward region though the correlation between the quark distributions
% and the jet cross sections increases with $x$ at high \pt. In summary,
% a significant reduction of the PDF uncertainties is expected by
% including the CMS inclusive jet cross section into fits of the proton
% structure.

\section{Setup of the HERAFitter fitting framework}
\label{section:herafittersetup}

% In the following the impact of the CMS inclusive jet data on proton
% PDFs is investigated by including the jet cross section measurement
% into a combined fit with the HERA inclusive DIS cross
% sections~\cite{Aaron:2009aa}, which were the basis for the
% determination of the HERAPDF1.0 PDF set.
%
% The analysis is performed within the HERAFitter framework, which is
% based on the Dokshitzer-Gribov-Lipatov-Altarelli-Parisi
% (DGLAP)~\cite{Gribov:1972ri,Altarelli:1977zs,Dokshitzer:1977sg}
% evolution scheme at NLO as implemented in the QCDNUM
% package~\cite{Botje:2010ay}. DIS data in the fit are required to have
% $Q^2 > Q_\mathrm{min}^2 = 3.5 \GeVsq$ to ensure the applicability of
% perturbative QCD\@. The following PDFs are assumed to be independent
% in the fit procedure: $xu_v(x)$, $xd_v(x)$, $xg(x)$, and
% $x\bar{U}(x)$, $x\bar{D}(x)$, where $x\bar{U}(x) = x\bar{u}(x)$, and
% $x\bar{D}(x) = x\bar{d}(x) + x\bar{s}(x)$. A parameterization with 13
% free parameters is used, similar to Ref.~\cite{Abramowicz:1900rp}. At
% the starting scale $Q_0$ of the QCD evolution, chosen to be $Q_0^2 =
% 1.9 \GeVsq$, the PDFs are parameterized as follows:
%
% \begin{align}
%   xg(x) &= A_g x^{B_g} (1-x)^{C_g} - A'_g x^{B'_g} (1-x)^{C'_g} \\
%   xu_v(x) &= A_{u_{v}} x^{B_{u_{v}}} (1-x)^{C_{u_{v}}} (1 + E_{u_{v}}x^2)\\
%   xd_v(x) &= A_{d_v} x^{B_{d_v}} (1-x)^{C_{d_{v}}}\\
%   x\bar U(x) &= A_{\bar U} x^{B_{\bar U}} (1-x)^{C_{\bar U}}\\
%   x\bar D(x) &= A_{\bar D} x^{B_{\bar D}} (1-x)^{C_{\bar D}}
% \end{align}
%
% The normalization parameters $A_g$, $A_{u_{v}}$ and $A_{d_{v}}$ are
% constrained by QCD sum rules. Additional constraints $B_{\bar
%   U}=B_{\bar D}$ and $A_{\bar U} = A_{\bar D}(1-f_s)$ are applied to
% ensure the same normalization for the $\bar u$ and $\bar d$ densities
% for $x \rightarrow 0$. The strangeness fraction is set to $f_s =
% 0.31$, as obtained from neutrino-induced di-muon
% production~\cite{Mason:2007zz}. The parameter $C'_g$ is fixed to
% 25~\cite{Martin:2009iq,Thorne:2006qt}. A generalized-mass variable
% flavour number scheme~\cite{Thorne:1997ga,Thorne:2006qt} is used with
% the strong coupling constant set to $\alpsmz= 0.1176$.


\subsection{Determination of PDF uncertainties in the HERAFitter framework}
\label{section:treatment_pdf_uncertainties}

% The uncertainty of the PDFs is subdivided into experimental, model,
% and parameterization uncertainties, which are studied
% separately. Experimental uncertainties result from the propagated
% statistical and systematic uncertainties of the input data.
% For the model uncertainties the following variations of
% model assumptions are considered:
%
% \begin{itemize}
% \item The strangeness fraction $f_s$, by default equal to $0.31$, is
%   varied between $0.23$ and $0.38$.
% \item The b-quark mass is varied by $\pm 0.25\GeV$ around the central
%   value of $4.75\GeV$.
% \item The minimum $Q^2$ value for data used in the fit,
%   $Q^2_\mathrm{min}$, is varied from $Q^2_\mathrm{min} = 3.5\GeVsq$ to
%   $5.0\GeVsq$. For the variation fit with $Q_{\mathrm{min}}^2 = 5.0
%   \GeVsq$, the parameter $B_{g}'$ is fixed to the value obtained in
%   the central fit.
% \end{itemize}
%
% The parameterization uncertainty is estimated as described in
% Ref.~\cite{Aaron:2009aa}. Employing the more general form of
% parameterization
%
% \begin{align}
%   xg(x) &= A_g x^{B_g} (1-x)^{C_g} (1  + D_g x + E_g x^2) - A'_g x^{B'_g} (1-x)^{C'_g};\\
%   xf(x) &= A_{f}  x^{B_{f}} (1-x)^{C_{f}} (1 + D_{f}x + E_{f}x^2)
% \end{align}
%
% it is tested successively whether the inclusion of additional fit
% parameters leads to a variation in the shape of the fitted results.
% Furthermore, the starting scale $Q_0$ is changed to $Q^2_0 =
% 1.5\GeVsq$ and $Q^2_0 = 2.5\GeVsq$.  The maximal deviations of the
% resulting PDFs from those obtained in the central fit define the
% parameterization uncertainty. The experimental, model, and
% parameterization uncertainties are added in quadrature to give the
% final PDF uncertainty.
%

\subsection{Definition of the goodness-of-fit estimator}
\label{sec:fitsetup}

% The agreement between data points $D_i$ and the theoretical
% predictions $T_i$ is estimated via a least-squares method. All $K$
% fully correlated sources of systematic uncertainty $\beta_{k}$ are
% treated using nuisance parameters $r_k$ and are assumed to be
% multiplicative to avoid the statistical bias that arises from
% uncertainty estimations taken from data~\cite{Lyons:1989gh}.
% % in order to avoid the d'Agostini bias~\cite{D'Agostini:2003nk}.
% As a consequence the covariance matrix,
% defined as $\mathrm{C} = \mathrm{cov}_{\mathrm{stat}} +
% \mathrm{cov}_{\mathrm{uncor}}$, has to be reevaluated in each
% iteration step.
%
% To inhibit the compensation of large systematic shifts by increasing
% the theory prediction and the statistical uncertainties at the same
% time, the systematic shifts of the theory are taken into account
% before the rescaling of the statistical uncertainty. Otherwise
% alternative minima in \chisq, which is defined as
%
% \begin{equation}
%   \chi^2 = \sum_{ij}^N \left(D_i - T_i - \sum_k^K r_k \beta_{ik}\right) \mathrm{C}_{ij}^{-1}
%   \left(D_j - T_j - \sum_k^K r_k \beta_{jk} \right) + \sum_k^K r_k^2\,,
%   \label{chi2_nuisance}
% \end{equation}
%
% could be found that are associated to a high theory prediction with at
% the same time large shifts (pulls) for the nuisance parameters. This
% is clearly undesirable~\cite{HERAFitter:2013hf}.


\subsection{Treatment of CMS data uncertainties}
\label{section:cmsdatauncertainties}

% The jet energy scale is the dominant source of experimental systematic
% uncertainty on jet cross sections. As described in
% Section~\ref{sec:measurementjec}, the \pt and $\eta$ dependent jet
% energy scale uncertainties are split into 16 uncorrelated sources
% which are fully correlated among themselves. Following the modified
% recommendation for the correlations versus rapidity of the
% \textsc{Single Pion} source as given in
% Section~\ref{sec:measurementjec}, it is necessary to split this source
% into five parts for the purpose of using the uncertainties published
% in~\cite{CMS-PAPERS-QCD-11-004} within the \chisq fits. The complete
% setup including other uncertainties is shown in
% Table~\ref{tab:cmsjets2011:nuisance}.
%
% Employing the technique of nuisance parameters, the impact of each
% systematic source of uncertainty on the fit result can be examined
% separately. In case of an adequate estimation of the size and the
% correlations of all uncertainties it is expected that the majority of
% all systematic sources are shifted by less than one standard deviation
% $\sigma$ from the default in the fitting procedure.
% Table~\ref{tab:cmsjets2011:nuisance} demonstrates that this is the
% case for the CMS jet data with the described procedure.
%
% \begin{table}[htbp]
%   \caption{19 independent sources of systematic uncertainty are considered in the
%     CMS inclusive jet measurement. Out of these 16 are related to the JES
%     and are listed first. In order to implement the improved correlation
%     treatment as described in Section~\ref{sec:measurementjec}, the
%     \textsc{SinglePion} source, see also the
%     Appendix~\ref{sec:jessources}, has been split up into five
%     sources. The shift from the default value in each source of systematic
%     uncertainty is determined by nuisance parameters in the fit and is
%     presented in units of standard deviations.}
%   \label{tab:cmsjets2011:nuisance}
%   \centering
%   \begin{tabular}{lrlr}
%     \hline\hline
%     Systematic source         & Shift in $\sigma$ & Systematic source & Shift in $\sigma$\rbthm\\\hline\hline
%     \textsc{Absolute}         & -0.20 &   \textsc{RelativeJEREC2}   & -0.28\rbtrr\\
%     \textsc{HighPtExtra}      & -0.60 &   \textsc{RelativeJERHF}    &  0.00\rbtrr\\
%     \textsc{SinglePionBarrel} &  0.64 &   \textsc{RelativeStatEC2}  & -0.28\rbtrr\\
%     \textsc{SinglePionEndcap} & -1.64 &   \textsc{RelativeStatHF}   &  0.00\rbtrr\\
%     \textsc{SinglePionY0005}  & -0.10 &   \textsc{RelativeFSR}      & -0.01\rbtrr\\
%     \textsc{SinglePionY0510}  &  0.12 &   \textsc{PileUpDataMC}     &  0.69\rbtrr\\
%     \textsc{SinglePionY1015}  &  0.85 &   \textsc{PileUpOOT}        & -0.09\rbtrr\\
%     \textsc{Flavor}           &  0.21 &   \textsc{PileUpPt}         &  0.48\rbtrr\\
%     \textsc{Time}             & -0.32 &   \textsc{PileUpBias}       &  0.26\rbtrr\\
%     \textsc{RelativeJEREC1}   &  0.55 &   \textsc{PileUpJetRate}    &  0.62\rbtrr\\
%     \hline
%     \textsc{Unfolding}        & -0.45 &   \textsc{Luminosity}       &  0.13\rbtrr\\
%     \textsc{NPCorrection}     &  0.84 &                             &      \rbtrr\\
%     \hline\hline
%   \end{tabular}
% \end{table}
%
% In contrast it was observed that with the original assumption of full
% correlation within the 16 JES systematic sources across all \yabs
% bins, tensions with shifts beyond $2\sigma$ became obvious and lead to
% a reexamination of this issue and the improved correlation treatment
% of the JES uncertainties as described
% previously in Section~\ref{sec:measurementjec}.



\section{Constraining PDFs with HERAFitter using the Dijet Measurement}
\label{section:cmsjets2011_pdfconstraints}

% The partial \chisq's per no.\ of data points \ndata are reported in
% Table~\ref{tab:fit:results} for each dataset in the HERA DIS or the
% combined fit including the CMS jet data. The achieved fit qualities
% demonstrate the compatibility of all data within the presented PDF
% fitting framework. The resulting parton distributions for the gluon,
% the up, the down, the sea, u valence, and d valence quarks with and
% without CMS jet data have been arranged next to each other in
% Figs.~\ref{fit:cmsjets2011:gud:fitscale}--\ref{fit:cmsjets2011:seauvdv:10000}.
% Figures~\ref{fit:cmsjets2011:gud:fitscale}
% and~\ref{fit:cmsjets2011:seauvdv:fitscale} present the effect at the
% starting scale of $Q^2 = 1.9 \GeVsq$, while in
% Figs.~\ref{fit:cmsjets2011:gud:10000}
% and~\ref{fit:cmsjets2011:seauvdv:10000} the results have been evolved
% to $Q^2 = 10^4 \GeVsq$.
%

% \begin{table}[htbp]
%   \caption{Partial \chisq's, \chipsq, for each dataset in the HERA DIS (middle
%     section) or the combined fit including CMS jet data (right section).
%     \ndata is the number of data points available for the determination of
%     the 13 parameters. The bottom two lines show the total \chisq and
%     \chisqndof. The difference between the sum of all
%     \chipsq and the total \chisq for the combined fit is attributed to
%     the nuisance parameters.}
%   \label{tab:fit:results}
%   \centering
%   \begin{tabular}{lr|rc|rc}
%     \hline\hline
%     \multicolumn{2}{c|}{} &
%     \multicolumn{2}{c|}{HERA data} &
%     \multicolumn{2}{c}{HERA \& CMS data}\rbthm\\
%     Dataset &
%     \multicolumn{1}{c|}{\ndata} &
%     \multicolumn{1}{c}{\chipsq} &
%     \multicolumn{1}{c|}{\chipsqndata} &
%     \multicolumn{1}{c}{\chipsq} &
%     \multicolumn{1}{c}{\chipsqndata}\rbthm\\\hline
%     NC HERA-I H1-ZEUS combined e-p. & 145 & 107 & 0.74 & 108 & 0.74 \rbtrr\\
%     NC HERA-I H1-ZEUS combined e+p. & 379 & 414 & 1.09 & 417 & 1.10 \rbtrr\\
%     CC HERA-I H1-ZEUS combined e-p. &  34 &  20 & 0.59 &  22 & 0.65 \rbtrr\\
%     CC HERA-I H1-ZEUS combined e+p. &  34 &  30 & 0.88 &  33 & 0.97 \rbtrr\\
%     CMS Inclusive Jets 2011         & 133 & --- &  --- & 107 & 0.80 \rbtrr\\\hline\hline
%     Dataset(s) & \ndof &
%     \multicolumn{1}{c}{\chisq} &
%     \multicolumn{1}{c|}{\chisqndof} &
%     \multicolumn{1}{c}{\chisq} &
%     \multicolumn{1}{c}{\chisqndof}\rbthm\\\hline
%     HERA data                       & 579 & 571 & 0.99 &    --- &  --- \rbtrr\\
%     HERA \& CMS data                & 712 &    --- &  --- & 694 & 0.97 \rbtrr\\
%     % All                             & 725 & 570.55 & 578 & 0.99 & 694.38 & 711 & 0.98 \rbtrr\\
%     \hline\hline
%   \end{tabular}
% \end{table}
%
%
% For a direct comparison
% Figs.~\ref{fit:cmsjets2011:directcomparison:fitscale}
% and~\ref{fit:cmsjets2011:directcomparison:10000} display for all PDFs
% at both scales, $Q^2 = 1.9 \GeVsq$ and $10^4 \GeVsq$, the fit results
% and total uncertainties with and without CMS jet data on top of each
% other. Finally, Fig.~\ref{fit:cmsjets2011:overview} shows an overview
% of the gluon, sea, u valence, and d valence distributions at the
% starting scale of $Q^2 = 1.9 \GeVsq$ within one plot.

% \begin{figure}[tbp]
%   \centering
%   \includegraphics[width=0.48\textwidth]{herafitter_Figures/cmsjets2011/HERADIS_14P_NLO_EIG_0_1_9}\hfill%
%   \includegraphics[width=0.48\textwidth]{herafitter_Figures/cmsjets2011/HERADISCMSJETS2011V2_14P_NLO_EIG_0_1_9}
%   \includegraphics[width=0.48\textwidth]{herafitter_Figures/cmsjets2011/HERADIS_14P_NLO_EIG_2_1_9}\hfill%
%   \includegraphics[width=0.48\textwidth]{herafitter_Figures/cmsjets2011/HERADISCMSJETS2011V2_14P_NLO_EIG_2_1_9}
%   \includegraphics[width=0.48\textwidth]{herafitter_Figures/cmsjets2011/HERADIS_14P_NLO_EIG_1_1_9}\hfill%
%   \includegraphics[width=0.48\textwidth]{herafitter_Figures/cmsjets2011/HERADISCMSJETS2011V2_14P_NLO_EIG_1_1_9}
%   \caption{The gluon (top), up quark (middle), and down quark (bottom)
%     PDFs as a function of $x$ as derived from HERA inclusive DIS data
%     alone (left) and in combination with CMS inclusive jet data from
%     2011 (right). The PDFs are shown at the starting scale $Q^2 = 1.9
%     \GeVsq$. The experimental (inner band), model (middle band), and
%     parameterization uncertainty (outer band) are successively added
%     quadratically to give the total uncertainty.}
%   \label{fit:cmsjets2011:gud:fitscale}
% \end{figure}
%
% \begin{figure}[tbp]
%   \centering
%   \includegraphics[width=0.48\textwidth]{herafitter_Figures/cmsjets2011/HERADIS_14P_NLO_EIG_9_1_9}\hfill%
%   \includegraphics[width=0.48\textwidth]{herafitter_Figures/cmsjets2011/HERADISCMSJETS2011V2_14P_NLO_EIG_9_1_9}
%   \includegraphics[width=0.48\textwidth]{herafitter_Figures/cmsjets2011/HERADIS_14P_NLO_EIG_8_1_9}\hfill%
%   \includegraphics[width=0.48\textwidth]{herafitter_Figures/cmsjets2011/HERADISCMSJETS2011V2_14P_NLO_EIG_8_1_9}
%   \includegraphics[width=0.48\textwidth]{herafitter_Figures/cmsjets2011/HERADIS_14P_NLO_EIG_7_1_9}\hfill%
%   \includegraphics[width=0.48\textwidth]{herafitter_Figures/cmsjets2011/HERADISCMSJETS2011V2_14P_NLO_EIG_7_1_9}
%   \caption{The sea quark (top), u valence quark (middle), and d
%     valence quark (bottom) PDFs as a function of $x$ as derived from
%     HERA inclusive DIS data alone (left) and in combination with CMS
%     inclusive jet data from 2011 (right). The PDFs are shown at the
%     starting scale $Q^2 = 1.9 \GeVsq$. The experimental (inner band),
%     model (middle band), and parameterization uncertainty (outer band)
%     are successively added quadratically to give the total
%     uncertainty.}
%   \label{fit:cmsjets2011:seauvdv:fitscale}
% \end{figure}
%
% \begin{figure}[tbp]
%   \centering
%   \includegraphics[width=0.48\textwidth]{herafitter_Figures/cmsjets2011/HERADIS_14P_NLO_EIG_0_10000_0}\hfill%
%   \includegraphics[width=0.48\textwidth]{herafitter_Figures/cmsjets2011/HERADISCMSJETS2011V2_14P_NLO_EIG_0_10000_0}
%   \includegraphics[width=0.48\textwidth]{herafitter_Figures/cmsjets2011/HERADIS_14P_NLO_EIG_2_10000_0}\hfill%
%   \includegraphics[width=0.48\textwidth]{herafitter_Figures/cmsjets2011/HERADISCMSJETS2011V2_14P_NLO_EIG_2_10000_0}
%   \includegraphics[width=0.48\textwidth]{herafitter_Figures/cmsjets2011/HERADIS_14P_NLO_EIG_1_10000_0}\hfill%
%   \includegraphics[width=0.48\textwidth]{herafitter_Figures/cmsjets2011/HERADISCMSJETS2011V2_14P_NLO_EIG_1_10000_0}
%   \caption{The gluon (top), up quark (middle), and down quark (bottom)
%     PDFs as a function of $x$ as derived from HERA inclusive DIS data
%     alone (left) and in combination with CMS inclusive jet data from
%     2011 (right). The PDFs are evolved to $Q^2 = 10^4 \GeVsq$. The
%     experimental (inner band), model (middle band), and
%     parameterization uncertainty (outer band) are successively added
%     quadratically to give the total uncertainty.}
%   \label{fit:cmsjets2011:gud:10000}
% \end{figure}
%
% \begin{figure}[tbp]
%   \centering
%   \includegraphics[width=0.48\textwidth]{herafitter_Figures/cmsjets2011/HERADIS_14P_NLO_EIG_9_10000_0}\hfill%
%   \includegraphics[width=0.48\textwidth]{herafitter_Figures/cmsjets2011/HERADISCMSJETS2011V2_14P_NLO_EIG_9_10000_0}
%   \includegraphics[width=0.48\textwidth]{herafitter_Figures/cmsjets2011/HERADIS_14P_NLO_EIG_8_10000_0}\hfill%
%   \includegraphics[width=0.48\textwidth]{herafitter_Figures/cmsjets2011/HERADISCMSJETS2011V2_14P_NLO_EIG_8_10000_0}
%   \includegraphics[width=0.48\textwidth]{herafitter_Figures/cmsjets2011/HERADIS_14P_NLO_EIG_7_10000_0}\hfill%
%   \includegraphics[width=0.48\textwidth]{herafitter_Figures/cmsjets2011/HERADISCMSJETS2011V2_14P_NLO_EIG_7_10000_0}
%   \caption{The sea quark (top), u valence quark (middle), and d
%     valence quark (bottom) PDFs as a function of $x$ as derived from
%     HERA inclusive DIS data alone (left) and in combination with CMS
%     inclusive jet data from 2011 (right). The PDFs are evolved to $Q^2
%     = 10^4 \GeVsq$. The experimental (inner band), model (middle
%     band), and parameterization uncertainty (outer band) are
%     successively added quadratically to give the total uncertainty.}
%   \label{fit:cmsjets2011:seauvdv:10000}
% \end{figure}
%
% \begin{figure}[tbp]
%   \centering
%   \includegraphics[width=0.48\textwidth]{herafitter_Figures/cmsjets2011/HERADIS_14P_NLO_DIRECT_EIG_0_1_9}\hfill%
%   \includegraphics[width=0.48\textwidth]{herafitter_Figures/cmsjets2011/HERADIS_14P_NLO_DIRECT_EIG_9_1_9}\hfill
%   \includegraphics[width=0.48\textwidth]{herafitter_Figures/cmsjets2011/HERADIS_14P_NLO_DIRECT_EIG_2_1_9}\hfill%
%   \includegraphics[width=0.48\textwidth]{herafitter_Figures/cmsjets2011/HERADIS_14P_NLO_DIRECT_EIG_8_1_9}\hfill
%   \includegraphics[width=0.48\textwidth]{herafitter_Figures/cmsjets2011/HERADIS_14P_NLO_DIRECT_EIG_1_1_9}\hfill%
%   \includegraphics[width=0.48\textwidth]{herafitter_Figures/cmsjets2011/HERADIS_14P_NLO_DIRECT_EIG_7_1_9}\hfill
%   \caption{The gluon (top left), sea quark (top right), u quark
%     (middle left), u valence quark (middle right), d quark (bottom
%     left) and the d valence quark (bottom right) PDFs as a function of
%     $x$ as derived from HERA inclusive DIS data alone (blue hatched) and in
%     combination with CMS inclusive jet data from 2011 (cyan).
%     The PDFs are shown at the starting scale $Q^2 = 1.9 \GeVsq$. Only
%     the total uncertainty of the PDFs is shown. }
%   \label{fit:cmsjets2011:directcomparison:fitscale}
% \end{figure}
%
% \begin{figure}[tbp]
%   \centering
%   \includegraphics[width=0.48\textwidth]{herafitter_Figures/cmsjets2011/HERADIS_14P_NLO_DIRECT_EIG_0_10000_0}\hfill%
%   \includegraphics[width=0.48\textwidth]{herafitter_Figures/cmsjets2011/HERADIS_14P_NLO_DIRECT_EIG_9_10000_0}\hfill
%   \includegraphics[width=0.48\textwidth]{herafitter_Figures/cmsjets2011/HERADIS_14P_NLO_DIRECT_EIG_2_10000_0}\hfill%
%   \includegraphics[width=0.48\textwidth]{herafitter_Figures/cmsjets2011/HERADIS_14P_NLO_DIRECT_EIG_8_10000_0}\hfill
%   \includegraphics[width=0.48\textwidth]{herafitter_Figures/cmsjets2011/HERADIS_14P_NLO_DIRECT_EIG_1_10000_0}\hfill%
%   \includegraphics[width=0.48\textwidth]{herafitter_Figures/cmsjets2011/HERADIS_14P_NLO_DIRECT_EIG_7_10000_0}\hfill
%   \caption{The gluon (top left), sea quark (top right), u quark
%     (middle left), u valence quark (middle right), d quark (bottom
%     left) and the d valence quark (bottom right) PDFs as a function of
%     $x$ as derived from HERA inclusive DIS data alone (blue hatched) and in
%     combination with CMS inclusive jet data from 2011 (cyan).
%     The PDFs are evolved to $Q^2 = 10^4 \GeVsq$. Only the total
%     uncertainty of the PDFs is shown.}
%   \label{fit:cmsjets2011:directcomparison:10000}
% \end{figure}
%
% \begin{figure}[tbp]
%   \centering
%   \includegraphics[width=\textwidth]{herafitter_Figures/cmsjets2011/HERADISCMSJETS2011V2QED_14P_NLO_OVERVIEW_EIG_1_9}
%   \caption{Overview of the gluon, sea, u valence, and d valence PDFs
%     before (full line) and after (dashed line) including the CMS jet
%     data into the fit. The PDFs are shown at the starting scale $Q^2 =
%     1.9 \GeVsq$. In addition the total uncertainty including the CMS
%     jet data is shown as a band around the central fit.}
%   \label{fit:cmsjets2011:overview}
% \end{figure}
%
% With respect to the gluon distribution, a significant improvement of
% precision in the high-$x$ region of $x \gtrsim 0.01$ is observed. In
% particular, the parameterization uncertainty is reduced.
% % For the gluon distribution it can be seen, as illustrated in the
% % figures, that uncertainties are significantly reduced in the high-$x$
% % region of $x \gtrsim 0.01$.
% % In the range of $0.03 < x < 0.3$ some increase in experimental
% % uncertainty is observed. This is not unusual, since previous
% % extrapolations implying large model and parameterization uncertainties
% % are now more restricted through the inclusion of the new data set with
% % its experimental uncertainties.
% % In particular the parameterization uncertainty is dramatically
% % improved.
% At the same time CMS jet data favour a larger gluon PDF at high $x$
% compared to the DIS data alone. As expected, no improvement is
% exhibited in the low-$x$ region, where the gluon is well constrained
% by the HERA data through scaling violations.
%
% It can also be seen that the parameterization and model uncertainties
% of the up- and down-quark distributions are reduced for $x >
% 0.3$. This is expected from the correlations, studied in
% Fig.~\ref{fig:correlation_pdf_xs_gqq}, where the quark distributions
% are constrained via the $qq$ contribution to jet production at high
% \yabs and \pt.
%
% Considering the significant increase in the u valence uncertainty
% exhibited in the low-$x$ region of
% Fig.~\ref{fit:cmsjets2011:directcomparison:fitscale} one has to
% remember that a more flexible parameterization is used than for the
% original HERAPDF1.0 fits and that in this region the contribution of
% the u valence quarks is very small. Through the QCD sum rules better
% constraints at high $x$ can then counterintuitively lead to larger
% uncertainties at low $x$, if this region is not sufficiently
% constrained by other data. In fact, the experimental uncertainty is
% reduced and only the model and parameterization uncertainties have
% become larger for this region, which is clearly visible from
% Fig.~\ref{fit:cmsjets2011:seauvdv:fitscale}.
%
% Inclusive DIS data alone are not sufficient to disentangle effects on
% cross section predictions from changes in the gluon distribution or
% the strong coupling constant at the same time. Therefore the strong
% coupling constant was always fixed to be $\alpsmz= 0.1176$ in all the
% presented results as in the original HERAPDF1.0 derivation. Including
% the CMS jet data this constraint can be dropped giving similar PDFs
% with, of course, larger uncertainties than previously. Direct
% comparisons of fits with and without the CMS data are not possible in
% this case, because the fitting procedure does not converge anymore
% with inclusive DIS data alone when \alpsmz is considered as a free
% parameter. %
% Including the CMS jet data the strong coupling constant is determined
% to be $\alpsmz = 0.1192\,^{+0.0017}_{-0.0015}$, where the uncertainty
% accounts for the experimental uncertainties of the inclusive HERA DIS
% and CMS jet data, and the NP uncertainties. The fitted value for
% \alpsmz is in agreement with the results described in the next
% section.
