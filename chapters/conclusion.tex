% vim: tw=80

\chapter{Conclusion}

With the LHC entering the precision era, it is crucial to have the best theory
predictions at hand for comparisons with measurements. The precision of the
proton PDFs plays an indispensable role, as the PDFs enter almost all cross
section predictions at the LHC. In particular the high $x$ region of the PDFs is
not yet well known. Since they are obtained from fits to experimental data, they
can be improved by considering LHC measurements in their determination.

In this thesis, a new dijet analysis has been developed. For the first time, a
triple-differential dijet cross section has been measured at the LHC. The
analysis was designed to provide constraints on the PDFs in the best possible
way. The cross sections were measured differentially as a function of the
average transverse momentum of the dijets, \ptavg, the boost of the dijet
system, \yboost, and the rapidity separation of the two jets, \ystar. In
particular the combination of these observables offers a high sensitivity to the
PDFs, especially in phase space regions containing boosted dijet events with
high transverse momenta.


\begin{figure}[h!tbp]
    \centering
    \includegraphics[width=0.47\textwidth]{figures/measurement/ptavg_spectrum.pdf}\hfill
    \includegraphics[width=0.45\textwidth]{figures/pdf_constraints/pdfcomp_direct_overview_1.9.pdf}
    \caption[Summary plot of results]{Left:
    The triple-differential dijet cross sections. The data are indicated by black
    markers, the NLO theory prediction by colored lines. Right: Overview of
    fitted PDFs with and without including the triple-differential dijet
    measurement.}
    \label{fig:conclusion}
\end{figure}

The measurement has been performed with the CMS detector at a center-of-mass
energy of \SI{8}{\TeV} using the complete data set recorded in 2012. Trigger,
reconstruction and detector effects have been carefully studied. The measured
cross sections have been corrected for detector effects in an iterative
unfolding procedure. The unfolded cross section have been compared with
pQCD prediction at NLO accuracy, see Fig.~\ref{fig:conclusion}. The data
is well described by theory over many orders of magnitude. In phase space
regions with highly boosted dijet events, the data discriminate between the
predictions using different global PDF sets because of the high experimental
precision. As a consequence, constraints on the PDFs can be provided. The
sensitivity of the PDFs is demonstrated by performing a PDF fit to DIS cross
sections of the HERA experiments and the dijet cross sections measured in this
thesis. It was found that the PDFs are improved if the dijet data are included
and the uncertainties of the PDFs, especially those of the gluon PDF, are
significantly reduced. Fig.~\ref{fig:conclusion} right demonstrates the PDF
constraints.

The strong coupling constant \asmz was determined in a simultaneous
fit together with the PDFs. The obtained value reads
%
\begin{equation*}
  \asmz = 0.1194_{-0.0015}^{+0.0015}(\mathrm{exp})_{-0.0002}^{+0.0002}(\mathrm{mod})_{-0.0004}^{+0.0002}(\mathrm{par})_{-0.0019}^{+0.0031}(\mathrm{scale})
\end{equation*}
%
and is in agreement with the world average value of $\asmz = 0.1181 \pm
0.0013$ determined by the PDG~\cite{Agashe:2014kda}. The dominant uncertainty is
of theoretical origin.

The pioneering studies presented in this thesis prove that the triple-differential
measurement of dijet cross sections using the chosen observables are an
excellent approach to perform QCD precision studies.

Two areas provide the opportunity for further improvements: In the near future,
NNLO corrections for dijet calculations will become available. The accuracy of the
cross section predictions will be improved and the determination of
\asmz will profit from reduced scale uncertainties as a consequence thereof. 

The restart of the LHC at the increased center-of-mass energy of $\SI{13}{\TeV}$
in combination with a higher instantaneous luminosity opens up a larger
accessible phase space. When CMS has collected a sufficiently large data sample,
it will be possible to extend this measurement to additional phase space regions
and improve the statistical precision.

\todo{Concluding sentence}
