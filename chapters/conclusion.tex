% vim: tw=80

\chapter{Conclusion}

With the LHC entering the precision era, it is crucial to also improve the
knowledge about the proton structure. As the PDFs are obtained from fits to
experimental data, further constraints can be derived by considering LHC
measurements in their determination.

In this thesis, a measurement of triple-differential dijet cross sections has
been performed for the first time at the LHC. It was designed in a
way to provide constraints on the PDFs in a best possible way. Up to now, the
uncertainties of the PDFs at high fractional momenta $x$ are sizable,
especially those of the gluon PDF. Because of the omnipresence of
PDFs at the LHC, many analyses would profit from improvements of the PDFs.
In particular searches for new physics at high energies, in which the highest
fractional momenta of the proton are probed, would benefit from reduced PDF
uncertainties.

These challenges were met with the development of a triple-differential dijet
analysis, in which the PDFs are probed over a wide range of $x$ and up to high
energy scales. Thus, the cross sections were measured differentially in the
average transverse momentum of the dijets, \ptavg, the boost of the dijet
system, \yboost, and the rapidity separation of the two jets, \ystar. It has
been shown, that these variables offer a high sensitivity to the PDFs,
especially in the region containing boosted dijet events.

\begin{figure}[h!tbp]
    \centering
    \includegraphics[width=0.47\textwidth]{figures/measurement/ptavg_spectrum.pdf}\hfill
    \includegraphics[width=0.45\textwidth]{figures/pdf_constraints/pdfcomp_direct_overview_1.9.pdf}
    \caption[Summary plot of results]{Left:
    The triple-differential dijet cross sections. The data points are indicated by black
    markers, the NLO theory prediction by colored lines. Right: Overview of
    fitted PDFs with and without including the triple-differential dijet
    measurement.}
    \label{fig:conclusion}
\end{figure}

The measurement has been performed using the CMS detector at a center-of-mass
energy of \SI{8}{\TeV} using the complete data set recorded in 2012 comprising
\SI{19.71}{\fbinv}. Thorough studies of trigger efficiencies and jet
identification ensure clean dijet events with a high selection efficiency.
Finally, the measured cross sections have been corrected for detector effects in
an iterative unfolding procedure to be able to compare with theory.

The theoretical predictions were calculated in pQCD at NLO accuracy and were
corrected for non-perturbative effects. Fig.~\ref{fig:conclusion} left shows the
cross sections in the six studied \ystar and \yboost bins. It was found that the
data is well described by the theory over many orders of magnitude. In
phase space regions containing highly boosted dijet events, the data
discriminate between the predictions using different global PDF sets because of
the high experimental precision.

As a consequence, constraints on the PDFs can be provided. The sensitivity of
the PDFs is demonstrated by performing a PDF fit to DIS cross sections of the
HERA experiments and the dijet cross sections measured in this thesis. It was
found that the PDFs are improved when the dijet data are included and the
uncertainties of the PDFs, especially those of the gluon PDF, can be
significantly reduced. Fig.~\ref{fig:conclusion} right demonstrates the PDF
constraints.

The strong coupling constant \asmz was determined by performing a simultaneous
fit of the PDFs and the strong coupling. The obtained value which reads
%
\begin{equation*}
  \asmz = 0.1188_{-0.0015}^{+0.0015}(\mathrm{exp})_{-0.0002}^{+0.0004}(\mathrm{mod})_{-0.0005}^{+0.0003}(\mathrm{par})_{-0.0010}^{+0.0029}(\mathrm{scale}),
\end{equation*}
%
is in good agreement with the world average value of the PDG. 

The pioneering studies presented in this thesis prove that the triple-differential
measurement of dijet cross sections using the chosen observables are an optimal
approach to perform QCD precision studies with dijets. The uncertainties on the
obtained value of \asmz are dominated by scale unceratinties. In view of
the upcoming NNLO dijet calculations, they should be revisited. The recent
restart of the LHC at $\sqrt{s}=\SI{13}{\TeV}$ opens up a larger accessible
phase space. When CMS has collected enough data, it will be possible to further
improve the precision and to study additional phase space regions.

\todo{Concluding sentence}
