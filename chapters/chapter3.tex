% vim: tw=80

\chapter{The CMS Detector at the Large Hadron Collider}

\todo{beschleunigergeschichte cern}


LEP adn Tevatron have provided remarkabable insights into the Standard Model and
precision tests of the Standard Model.

\section{The Large Hadron Collider}

The Large Hadron Collider (LHC) is the world's most powerful particle
accelerator and collider. The LHC is contained in the circular tunnel of the
preceding Large Electron Positron (LEP) collider which has a circumference of 27
km at a depth between 50 and 175 m. The tunnel crosses the border between
Switzerland and France at four points and two of the four main experiments are
located in Frace. 

Two adjacent beamlines that intersect at four interactions points contain the
particle beams travelling in opposite directions. More than 1000 dipole magnets,
generating a magnetic field of up to 8.3 T bend the proton beams on the circular track while
almost 400 quadrupole magnets keep the beams focused. 

The proton beams are brought to collision at four interaction points which house
the LHC experiments ALICE~\cite{ALICE}, ATLAS~\cite{ATLASa}, CMS and
LHCb~\cite{LHCb}. ALICE is designed to study the quark-gluon plasma produced by
colliding heavy ions, which resembles the initial state of the universe. LHCB is
precisely measuring the CP violation and the decay of B mesons. ATLAS and CMS,
general purpose detectors allowing a broad field of phycis studies, were built
to search and study the Higgs boson and physics models beyond the standard
model. Additionally precision measurements of standard model predictions and
parameters improve the current knowledge and the confidence of predictions of
the standard model.

Prior to the injection and acceleration of protons in the LHC, the particles
pass a series of consecutive acceleration steps succesively increasing their
energy. The linear particle accelerator (LINAC2) generates 50 MeV protons, that
are further accelerated in the Proton-Synchrotron (PS) to 26 GeV and
the Super-Proton-Synchrotron (SPS) up to an energy of \SI{450}{\giga
\electronvolt}. At last the proton beams are injected into the LHC ring and are
further accelerated up to peak design energy of 13 TeV. All these
pre-accelerators are not only used to feed the LHC, but also serve other physic
experiments as can be seen in Figure~\ref{fig:lhc_complex}.

\begin{figure}[htb]
    \centering
    \includegraphics[width=0.8\textwidth]{figures/cms_detector/lhc_accelerator_chain.pdf}
    \caption[CERN accelerator complex]{Schematic view of the CERN
        accelerator complex. It shows the LHC ring  with the four beam crossing
        points as well as the various pre-accelerators~\cite{LHC:COMPLEX}.}
        \label{fig:lhc_complex}
\end{figure}

\subsection{The Compact Muon Solenoid Detector}

The Compact Muon Solenoid (CMS) detector is a general purpose detector at the
LHC located at Point 5, at the opposite side of the CERN campus at Meyrin. To
serve a vast range of physics studies, its design is driven by a cylyndrical
structure containing layers of different subdetectors each built
to measure a specific type of particles at high precision. A high-precision
inner tracking system is surrounded by an electromagnetic and hadronic
calorimeter which are encased by a superconducting solenoid magnet, and a high
precision muon detection system. The detector is 21.6 m long and 14.6 m in
diameter, weighing more than 12000 tons due to its compact design. The detector
was built as cylyndrical slices constructed at ground level and lowered into the
cavern. In case of upgrades or repairs, the slices can be pulled apart and easy
access to the inner components is gained. 

\subsection{Definition of the Coordinate System}

CMS uses a right-handed coordinate system centered at the nominal interaction point inside the
detector. The $x$-axis points radially inward towards the LHC ring centre, the
$y$-axis vertically upwards and the $z$-axis along the beam direction towards
the Jura mountains. Important quantities are the azimuthal angle $\phi$
measured from the $x$-axis in the $x$-$y$ plane and the polar angle $\theta$,
measured from the $z$-axis in the $z$-$y$ plane. Instead of the polar angle
$theta$, the pseudo-rapidity $\eta$ and the rapidity $y$ are commonly used to
split the phasespace, since the differential flux is approximately constant at
hadron-hadron colliders. The pseudo-rapidity is defined as

\begin{equation}
    \eta = - \ln \left( \tan \left( \frac{\theta}{2} \right) \right)
\end{equation}

Throughout this thesis the rapidity is favoured compared to the pseudo-rapidity
due to its invariance under longitudinal boosts in the $z$-direction. Rapidity
and pseudo-rapidity are equivalent in the case of massless particle. The
rapidity is defined as

\begin{equation}
    y = \frac{1}{2} \ln \left( \frac{E + p_z}{E - p_z} \right) 
\end{equation}

The momentum along the beamline is not well defined due to the momentum
distribution inside the proton. A direct connection to the hard process is given
by the transverse momentum \pt related to cartesian coordinates as

\begin{equation}
    \pt = \sqrt{p_x^2 + p_y^2}
\end{equation}

\subsection{Inner Tracking System}

To yield a best possible spatial resolution, the particle tracks needs be
measured as close to the beam line as possible. The inner tracking system of CMS
is designed to measure the tracks of charged particles emerging from the
collision. Enclosing the interaction point with its diameter of 2.6 m and
extending 2.8 m in each direction of the beamline, the tracking system covers a
pseudo-rapidity range up to $|\eta| < 2.5$. 

The inner tracking system comprises two subsystems, the silicon pixel detector,
consisting of three layers and installed very close to the beam pipe, and the
silicon strip detector with ten layers in the barrel region. The pixel detector
contains 65 millon pixels arranged in three cylindrical layers at 4 cm, 7 cm and
11 cm to the beam pipe. The pixel detector is able to resolve the huge number of
particle tracks of around 10 million particles per square centimetre and
reconstruct their tracks. The high spatial resolution achieved by the pixel
detector additionally allows the identification and measurement of secondary
vertices used to identify long-lived particles.

Reduced particle flux further away from the beam pipe eases the identification
of tracks. Cost-efficient silicon strip detectors are employed reaching out to
an radius of 1.3 m. The strip detector consists of a total of 10 million
detecting strips read out by 80,000 chips. 

\begin{figure}[htb]
    \centering
    \includegraphics[width=0.65\textwidth]{figures/cms_detector/tracker.pdf}\hfill
    \includegraphics[width=0.3\textwidth]{figures/cms_detector/tracking_sytem_barrel_slice.png}
    \caption[Inner Tracking System]{The left figure shows one quadrant of the
    longitudinal section of the inner tracking system of CMS consisting of the
silicon pixel detector and the silicon strip detector. The right figure shows a
transverse section of the tracking system in the barrel region with overlapping
arrangement of the strip modules. Figures from~\cite{phd:joramberger} and
\cite{cmsweb:innertracker}.}
        \label{fig:cms:inner_tracking}
\end{figure}

\subsection{Electromagnetic Calorimeter}
\subsection{Hadronic Calorimeter}
%thin
% particle flow needed to improve jet resolution
% strong magnetic field is the key (Atlas 2T vs. CMS 4T
\subsection{Superconducting Solenoid}
\subsection{Muon System}

\subsection{Data aquisition}

\subsection{Trigger system}

\section{Software and Computing Infrastructure}

% cmssw grid-control, root,

% physics software: herwig/pythia nlojet++/fastNLO:
